\documentclass[class=NCU_thesis, crop=false]{standalone}
\begin{document}

\chapter{實驗設計與結果}

\section{臉部辨識實驗}
\subsection{模型訓練結果}
準確率達98\%,
詳細訓練結果請見\cref{fig:fig-result-face}。
\fig[0.8][fig:fig-result-face][!hbt]{fig-result-face.png}[臉部辨識訓練結果][臉部辨識訓練結果]

\subsection{實驗評估方式}
臉部辨識實驗評估方式 臉部辨識實驗評估方式

\subsection{實驗結果分析}
臉部辨識實驗結果分析 臉部辨識實驗結果分析

\section{奶嘴辨識實驗}
\subsection{模型訓練結果}
準確率達98\%,
詳細訓練結果請見\cref{fig:fig-result-pacifier}。
\fig[0.8][fig:fig-result-pacifier][!hbt]{fig-result-pacifier.png}[奶嘴辨識訓練結果][奶嘴辨識訓練結果]

\subsection{實驗設計}
奶嘴辨識實驗實驗設計 奶嘴辨識實驗實驗設計

\subsection{實驗評估方式}
奶嘴辨識實驗實驗評估方式 奶嘴辨識實驗實驗評估方式

\subsection{實驗結果分析}
奶嘴辨識實驗實驗結果分析 奶嘴辨識實驗實驗結果分析

\section{臉部遮擋辨識實驗}
\subsection{實驗設計}
奶嘴辨識實驗實驗設計 奶嘴辨識實驗實驗設計

\subsection{實驗評估方式}
奶嘴辨識實驗實驗評估方式 奶嘴辨識實驗實驗評估方式

\subsection{實驗結果分析}
首先,使用503張影像測試臉部遮擋辨識,
其中僅一張影像(
\cref{fig:fig-face-pacifier-error}。
\fig[0.4][fig:fig-face-pacifier-error][!hbt]{fig-face-pacifier-error.jpg}[誤判影像:真實類別為安全,但預測為警示][誤判影像:真實類別為安全,但預測為警示]
)類別判斷錯誤,
誤將安全狀態辨識為警示狀態。
推測原因為該張影像畫質較差,面部影像不清而導致誤判。

再者,我們使用232張影像測試奶嘴辨識,
所有測試集皆辨識正確。

由於此二模型測試正確率高,故本文中不放入其混淆矩陣。

\section{姿勢分類實驗}
\subsection{模型訓練結果}
準確率達99\%,
詳細訓練結果請見\cref{fig:fig-result-four-classes}。
\fig[0.8][fig:fig-result-four-classes][!hbt]{fig-result-four-classes.png}[姿勢辨識訓練結果][姿勢辨識訓練結果]

\subsection{實驗設計}
姿勢分類實驗設計 姿勢分類實驗設計

\subsection{實驗評估方式}
姿勢分類實驗評估方式 姿勢分類實驗評估方式

\subsection{實驗結果分析}
我們使用744張影像針對此模型進行測試,
包含了五張類別辨識錯誤的影像,
其中有三張將坐姿誤判為趴躺姿勢,
推測原因為嬰兒雖呈現坐姿,
但上半身貼近其腿部,而導致誤判。

此模型之混淆矩陣,請見\cref{fig:fig-confusion-matrix-four-classes}。
\fig[1.1][fig:fig-confusion-matrix-four-classes][!hbt]{fig-confusion-matrix-four-classes.png}[嬰兒姿勢辨識之混淆矩陣][嬰兒姿勢辨識之混淆矩陣]

\section{影片危險偵測實驗}
\subsection{實驗設計}
影片危險偵測實驗 影片危險偵測實驗

\subsection{實驗評估方式}
影片危險偵測實驗 影片危險偵測實驗

\subsection{實驗結果分析}
影片危險偵測實驗 影片危險偵測實驗

\end{document}