\documentclass[class=NCU_thesis, crop=false]{standalone}
\begin{document}

\chapter{實驗設計與結果}

\section{嬰兒臉部偵測實驗}
\subsection{實驗目的與設計}
在收集嬰兒臉部資料集時,
需針對嬰兒影像擷取出臉部範圍,
進而後續之臉部遮擋辨識階段。

因此,本實驗使用3.3節之嬰兒姿勢資料集,
分別以精確度較低但執行時間較短之
OpenCV~\cite{goyal_face_2017}
與SSD~\cite{ye_face_2021}
以及精確度較高但需較長偵測時間之
MTCNN~\cite{xiang_joint_2017}
與RetinaFace~\cite{deng_retinaface_2020}
等臉部偵測演算法,
根據其臉部擷取效果及執行速度進行比較,
以驗證適合本系統之演算法。

\subsection{實驗評估方式}
本實驗為驗證嬰兒臉部偵測演算法之實際可行性,
將針對執行時間及準確度分別進行比較,
透過計算演算法偵測整個資料集共15416張影像所花費之時間,
得以算出各演算法平均每張需花費之時間;
而準確度則將嬰兒臉部偵測之影像結果進行標示,
分別計算出各演算法之accuracy、precision及recall。

\subsection{實驗結果與分析}
在本實驗中,
第一部分針對演算法之執行時間,
以OpenCV及SSD兩演算法進行比較:
使用OpenCV演算法偵測15416張影像共花費xx,
亦即平均每張影像需花xx秒;
而使用SSD演算法偵測15416張影像則共花費xx,
即平均每張影像需花xx秒。
但又因OpenCV在準確度及召回率皆過低,
無法使用於系統中,
故我們使用SSD進行初步之嬰兒臉部偵測。

而本實驗之第二部分則就精確度進行MTCNN與RetinaFace兩演算法進行比較:
使用MTCNN演算法進行嬰兒臉部偵測時,
偵測結果如\cref{table:table-mtcnn},
其accuracy為90.20\%、precision為94.76\%以及recall為90.93\%;
而使用RetinaFace演算法進行嬰兒臉部偵測時,
影像偵測結果如\cref{table:table-retinaface},
其accuracy為99.78\%、precision為99.75\%以及recall為99.91\%。
因此,
本系統為達到能更高的正確率及召回率,
在SSD找不到臉部時,
將接續使用了RetinaFace演算法進行嬰兒臉部偵測。

\begin{table}[h]
    \centering
    \caption{MTCNN演算法偵測嬰兒臉部結果}
    \label{table:table-mtcnn}
    \begin{tabular}{ccc}
    \hline
     & True(預測有臉) & False(預測無臉) \\
    \hline
    True & \multirow{2}{*}{9361} & \multirow{2}{*}{994} \\
    (偵測正確或實際有臉)  & & \\
    False & \multirow{2}{*}{517} & \multirow{2}{*}{4544} \\
    (偵測錯誤或實際無臉) &  & \\
    \hline
    Total & \textbf{9878} & \textbf{5538} \\
    \hline
    \end{tabular}
\end{table}

\begin{table}[h]
    \centering
    \caption{RetinaFace演算法偵測嬰兒臉部結果}
    \label{table:table-retinaface}
    \begin{tabular}{ccc}
    \hline
     & True(預測有臉) & False(預測無臉) \\
    \hline
    True & \multirow{2}{*}{12925} & \multirow{2}{*}{11} \\
    (偵測正確或實際有臉)  & & \\
    False & \multirow{2}{*}{33} & \multirow{2}{*}{2447} \\
    (偵測錯誤或實際無臉) &  & \\
    \hline
    Total & \textbf{12958} & \textbf{2458} \\
    \hline
    \end{tabular}
\end{table}

\section{臉部遮擋辨識實驗}
\subsection{模型訓練結果}
準確率達98\%,
詳細訓練結果請見\cref{fig:fig-result-face}。
\fig[0.8][fig:fig-result-face][!hbt]{fig-result-face.png}[臉部辨識訓練結果][臉部辨識訓練結果]

\subsection{實驗設計}
奶嘴辨識實驗實驗設計 奶嘴辨識實驗實驗設計

\subsection{實驗評估方式}
奶嘴辨識實驗實驗評估方式 奶嘴辨識實驗實驗評估方式

\subsection{實驗結果分析}
首先,使用503張影像測試臉部遮擋辨識,
其中僅一張影像(
\cref{fig:fig-face-pacifier-error}。
\fig[0.4][fig:fig-face-pacifier-error][!hbt]{fig-face-pacifier-error.jpg}[誤判影像:真實類別為安全,但預測為警示][誤判影像:真實類別為安全,但預測為警示]
)類別判斷錯誤,
誤將安全狀態辨識為警示狀態。
推測原因為該張影像畫質較差,面部影像不清而導致誤判。

再者,我們使用232張影像測試奶嘴辨識,
所有測試集皆辨識正確。

由於此二模型測試正確率高,故本文中不放入其混淆矩陣。

\section{姿勢分類實驗}
\subsection{模型訓練結果}
準確率達99\%,
詳細訓練結果請見\cref{fig:fig-result-four-classes}。
\fig[0.8][fig:fig-result-four-classes][!hbt]{fig-result-four-classes.png}[姿勢辨識訓練結果][姿勢辨識訓練結果]

\subsection{實驗設計}
姿勢分類實驗設計 姿勢分類實驗設計

\subsection{實驗評估方式}
姿勢分類實驗評估方式 姿勢分類實驗評估方式

\subsection{實驗結果分析}
我們使用744張影像針對此模型進行測試,
包含了五張類別辨識錯誤的影像,
其中有三張將坐姿誤判為趴躺姿勢,
推測原因為嬰兒雖呈現坐姿,
但上半身貼近其腿部,而導致誤判。

此模型之混淆矩陣,請見\cref{fig:fig-confusion-matrix-four-classes}。
\fig[1.1][fig:fig-confusion-matrix-four-classes][!hbt]{fig-confusion-matrix-four-classes.png}[嬰兒姿勢辨識之混淆矩陣][嬰兒姿勢辨識之混淆矩陣]

\section{影片危險偵測實驗}
\subsection{實驗設計}
影片危險偵測實驗 影片危險偵測實驗

\subsection{實驗評估方式}
影片危險偵測實驗 影片危險偵測實驗

\subsection{實驗結果分析}
影片危險偵測實驗 影片危險偵測實驗

\end{document}