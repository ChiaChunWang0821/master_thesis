\documentclass[class=NCU_thesis, crop=false]{standalone}
\usepackage{multirow}

\begin{document}

\chapter{實驗設計與結果}
本章節根據第三章之研究方法,
設計以下五項實驗以驗證本系統之設計:
臉部偵測準確度、臉部偵測執行時間、臉部遮擋辨識、姿勢辨識及影片危險偵測等,
並於此章各節詳盡介紹實驗之目的與設計、評估方式以及結果與分析。

另外,
本章各項實驗皆於相同硬體環境操作,詳細資訊如下:
\begin{itemize}
    \item 作業系統:Windows 10
    \item CPU:Intel(R) Core(TM) i7-10700KF CPU @ 3.80GHz
    \item 記憶體:128GB
    \item GPU:NVIDIA GeForce GTX 1660
\end{itemize}

\section{臉部偵測準確度實驗}
\subsection{實驗目的與設計}
在收集嬰兒臉部資料集時,
需針對嬰兒影像擷取出臉部範圍,
進而後續之臉部遮擋辨識階段。

為了使本系統擁有較佳的臉部偵測準確性且兼具執行效能,
本文透過4.1節及4.2節之實驗,
分別進行臉部偵測演算法準確度與執行時間之比較,
以驗證以下設計:
先使用SSD演算法偵測嬰兒臉部,
此方法雖常未能如期找到嬰兒臉部範圍,
但其準確度很高,
故能利用此算法之時間優勢;
而為補足SSD演算法召回率不足之缺失,
若其找不到嬰兒臉部時,
則接續使用RetinaFace演算法,
利用其正確率及準確率皆高之優點進行嬰兒臉部偵測。

故本實驗使用3.3.1節之嬰兒姿勢資料集,
分析OpenCV~\cite{goyal_face_2017}
、SSD~\cite{ye_face_2021}
、MTCNN~\cite{zhang_joint_2016}
及RetinaFace~\cite{deng_retinaface_2020}
等臉部偵測演算法,
計算其臉部擷取之準確度。

\subsection{實驗評估方式}
本實驗針對四項演算法之準確度進行比較,
將嬰兒臉部偵測結果影像進行分類標註,
以計算出各演算法之accuracy、precision及recall。

\subsection{實驗結果與分析}
由\cref{table:table-mtcnn}及\cref{table:table-retinaface}
可見MTCNN與RetinaFace之實驗結果。
而\cref{table:table-opencv}及\cref{table:table-ssd}
為OpenCV與SSD之實驗結果,
可發現其將多數影像皆誤判為無臉(False),
亦即影像中有嬰兒臉部畫面但演算法未偵測之,
故此部分僅關注判斷為有臉(True)之數據統計。
\begin{table}[h]
    \centering
    \caption{MTCNN~\cite{zhang_joint_2016}偵測嬰兒臉部結果}
    \label{table:table-mtcnn}
    \begin{tabular}{ccc}
    \hline
     & True(預測有臉)& False(預測無臉)\\
    \hline
    True & \multirow{2}{*}{9361} & \multirow{2}{*}{994} \\
    (實際有臉)& & \\
    False & \multirow{2}{*}{517} & \multirow{2}{*}{4544} \\
    (實際無臉)&  & \\
    \hline
    Total & \textbf{9878} & \textbf{5538} \\
    \hline
    \end{tabular}
\end{table}

\begin{table}[h]
    \centering
    \caption{RetinaFace~\cite{deng_retinaface_2020}偵測嬰兒臉部結果}
    \label{table:table-retinaface}
    \begin{tabular}{ccc}
    \hline
     & True(預測有臉)& False(預測無臉)\\
    \hline
    True & \multirow{2}{*}{12925} & \multirow{2}{*}{11} \\
    (實際有臉)& & \\
    False & \multirow{2}{*}{33} & \multirow{2}{*}{2447} \\
    (實際無臉)&  & \\
    \hline
    Total & \textbf{12958} & \textbf{2458} \\
    \hline
    \end{tabular}
\end{table}

\begin{table}[h]
    \centering
    \caption{OpenCV~\cite{goyal_face_2017}偵測嬰兒臉部結果}
    \label{table:table-opencv}
    \begin{tabular}{ccc}
    \hline
     & True(預測有臉)& False(預測無臉)\\
    \hline
    True & \multirow{2}{*}{2882} & \multirow{4}{*}{11809} \\
    (實際有臉)& & \\
    False & \multirow{2}{*}{725} & \\
    (實際無臉)&  & \\
    \hline
    \end{tabular}
\end{table}

\begin{table}[h]
    \centering
    \caption{SSD~\cite{ye_face_2021}偵測嬰兒臉部結果}
    \label{table:table-ssd}
    \begin{tabular}{ccc}
    \hline
     & True(預測有臉)& False(預測無臉)\\
    \hline
    True & \multirow{2}{*}{4830} & \multirow{4}{*}{10581} \\
    (實際有臉)& & \\
    False & \multirow{2}{*}{5} & \\
    (實際無臉)&  & \\
    \hline
    \end{tabular}
\end{table}

再經計算後,
四項演算法之accuracy、precision及recall值如下:
\begin{enumerate}
    \item OpenCV:由於多數影像皆誤判為無臉(False),故僅計算其precision為79.90\%。
    \item SSD:由於多數影像皆誤判為無臉(False),故僅計算其precision為99.90\%。
    \item MTCNN:accuracy為90.20\%、precision為94.76\%以及recall為90.93\%。
    \item RetinaFace:accuracy為99.78\%、precision為99.75\%以及recall為99.91\%。
\end{enumerate}

因此,
透過本實驗結果可得出選用RetinaFace演算法行嬰兒臉部偵測,
可擁有較佳的偵測準確度。

\section{臉部偵測執行時間實驗}
\subsection{實驗目的與設計}
本研究進行嬰兒臉部偵測除了考量準確度外,
亦希望提升整體系統之執行效率。

故本實驗使用3.3.1節之嬰兒姿勢資料集,
分析OpenCV~\cite{goyal_face_2017}
、SSD~\cite{ye_face_2021}
、MTCNN~\cite{zhang_joint_2016}
及RetinaFace~\cite{deng_retinaface_2020}
等臉部偵測演算法之執行時間,
以驗證適合本系統之演算法。

\subsection{實驗評估方式}
本實驗針對四項演算法之執行速度進行比較,
透過計算演算法偵測15416張影像資料集所花費之時間,
計算各演算法平均偵測一張影像之執行時間。

\subsection{實驗結果與分析}
OpenCV、SSD、MTCNN及RetinaFace四項演算法偵測15416張影像之詳細實驗結果如下:
\begin{enumerate}
    \item OpenCV:共18分01.78秒,平均每張影像需0.07秒。
    \item SSD:共9分17.26秒,平均每張影像需0.04秒。
    \item MTCNN:共2小時8分22.05秒,平均每張影像需0.50秒。
    \item RetinaFace:共5小時42分2.10秒,平均每張影像需1.33秒。
\end{enumerate}

因此,
透過本實驗結果可得出使用SSD演算法進行嬰兒臉部偵測,
將可擁有較佳的偵測速度。

總結4.1節與4.2節之實驗結果,
驗證本系統先使用SSD演算法偵測嬰兒臉部,
未如期找到目標時,
則改以RetinaFace演算法偵測,
達成兼具準確性及執行效率之系統目標。

\section{臉部遮擋辨識實驗}
\subsection{實驗目的與設計}
本系統為偵測嬰兒臉部是否遭非奶嘴之異物遮擋,
使用3.2.2節之嬰兒臉部資料集以ResNet50~\cite{he_deep_2016}訓練模型,
並透過驗證集進行模型驗證。

程式實作中,
網路訓練回合數為20,
設定影像資料大小為224x224,
包含三個類別(臉部無遮擋之安全狀態、臉部遭奶嘴遮擋及臉部遭異物遮擋之危險狀態),
且透過data augmentation技術生成訓練及測試資料,
輸出層使用softmax作為激發函數,
並使用Adam作為optimizer且將學習率設為0.000001以進行微調。

\subsection{實驗結果分析}
本實驗訓練之模型其最終訓練準確率達98.06\%,
而測試準確率達99.43\%,
詳細訓練結果請見\cref{fig:fig-result-face-occlusion}。
\fig[0.9][fig:fig-result-face-occlusion][!hbt]{fig-result-face-occlusion.png}[臉部辨識訓練結果][臉部辨識訓練結果]

接著,再使用342張之驗證集影像進行模型驗證,
所有影像皆辨識正確,
其混淆矩陣如\cref{table:table-confusion-matrix-face-occlusion},
表中數字為各類影像之張數及占比。
\begin{table}[h]
    \centering
    \caption{臉部遮擋辨識模型之混淆矩陣}
    \label{table:table-confusion-matrix-face-occlusion}
    \begin{tabular}{|cc|ccc|}
        \hline
        \multicolumn{2}{|l|}{} & \multicolumn{3}{c|}{\textbf{預測分類}} \\ \cline{3-5} 
        \multicolumn{2}{|l|}{\multirow{-2}{*}{}} & \multicolumn{1}{c|}{\textbf{無遮蔽}} & \multicolumn{1}{c|}{\textbf{使用奶嘴}} & \textbf{有異物} \\ \hline
        \multicolumn{1}{|c|}{} & \textbf{無遮蔽}  & \multicolumn{1}{c|}{{\color[HTML]{FE0000} \begin{tabular}[c]{@{}c@{}}120\\ (1.00)\end{tabular}}} & \multicolumn{1}{c|}{\begin{tabular}[c]{@{}c@{}}0\\ (0.00)\end{tabular}} & \begin{tabular}[c]{@{}c@{}}0\\ (0.00)\end{tabular} \\ \cline{2-5} 
        \multicolumn{1}{|c|}{\multirow{-3}{*}{\textbf{實際分類}}} & \textbf{使用奶嘴} & \multicolumn{1}{c|}{\begin{tabular}[c]{@{}c@{}}0\\ (0.00)\end{tabular}} & \multicolumn{1}{c|}{{\color[HTML]{FE0000} \begin{tabular}[c]{@{}c@{}}115\\ (1.00)\end{tabular}}} & \begin{tabular}[c]{@{}c@{}}0\\ (0.00)\end{tabular} \\ \cline{2-5} 
        \multicolumn{1}{|c|}{} & \textbf{有異物} & \multicolumn{1}{c|}{\begin{tabular}[c]{@{}c@{}}0\\ (0.00)\end{tabular}} & \multicolumn{1}{c|}{\begin{tabular}[c]{@{}c@{}}0\\ (0.00)\end{tabular}} & {\color[HTML]{FE0000} \begin{tabular}[c]{@{}c@{}}107\\ (1.00)\end{tabular}} \\ \hline
    \end{tabular}
\end{table}

\section{姿勢辨識實驗}
\subsection{實驗目的與設計}
本系統為辨識嬰兒姿勢是否處於危險狀態,
使用3.3.1節之資料集以ResNet50~\cite{he_deep_2016}訓練模型,
並透過驗證集進行模型驗證。

程式實作中,
網路訓練回合數為20,
設定影像資料大小為224x224,
包含四個類別(正躺、趴躺、坐姿及站立),
且透過data augmentation技術生成訓練及測試資料,
輸出層使用softmax作為激發函數,
並使用Adam作為optimizer且將學習率設為0.000001以進行微調。

\subsection{實驗結果分析}
本實驗訓練之模型其最終訓練準確率達99.45\%,
而測試準確率達99.71\%,
詳細訓練結果請見\cref{fig:fig-result-four-classes}。
\fig[0.9][fig:fig-result-four-classes][!hbt]{fig-result-four-classes.png}[姿勢辨識訓練結果][姿勢辨識訓練結果]

接著,再使用744張之驗證集進行模型驗證,
包含了五張類別辨識錯誤的影像,
其中三張將坐姿誤判為趴躺姿勢,
推測原因為嬰兒雖呈現坐姿,
但上半身貼近其腿部(如\cref{fig:fig-error-four-classes}),
而導致誤判。
此模型之混淆矩陣如\cref{table:table-confusion-matrix-four-classes},
表中數字為各類影像之張數及占比。
\fig[0.6][fig:fig-error-four-classes][!hbt]{fig-error-four-classes.jpg}[姿勢辨識錯誤之影像:坐姿誤判為趴躺]

\begin{table}[h]
    \centering
    \caption{姿勢辨識模型之混淆矩陣}
    \label{table:table-confusion-matrix-four-classes}
    \begin{tabular}{|cc|cccc|}
        \hline
        \multicolumn{2}{|l|}{} & \multicolumn{4}{c|}{\textbf{預測分類}} \\ \cline{3-6} 
        \multicolumn{2}{|l|}{\multirow{-2}{*}{}} & \multicolumn{1}{c|}{\textbf{正躺}} & \multicolumn{1}{c|}{\textbf{趴躺}} & \multicolumn{1}{c|}{\textbf{坐姿}} & \multicolumn{1}{l|}{\textbf{站立}} \\ \hline
        \multicolumn{1}{|c|}{} & \textbf{正躺} & \multicolumn{1}{c|}{{\color[HTML]{FE0000} \begin{tabular}[c]{@{}c@{}}164\\ (1.00)\end{tabular}}} & \multicolumn{1}{c|}{\begin{tabular}[c]{@{}c@{}}0\\ (0.00)\end{tabular}} & \multicolumn{1}{c|}{\begin{tabular}[c]{@{}c@{}}0\\ (0.00)\end{tabular}} & \begin{tabular}[c]{@{}c@{}}0\\ (0.00)\end{tabular} \\ \cline{2-6} 
        \multicolumn{1}{|c|}{} & \textbf{趴躺} & \multicolumn{1}{c|}{\begin{tabular}[c]{@{}c@{}}1\\ (0.01)\end{tabular}} & \multicolumn{1}{c|}{{\color[HTML]{FE0000} \begin{tabular}[c]{@{}c@{}}191\\ (0.99)\end{tabular}}} & \multicolumn{1}{c|}{\begin{tabular}[c]{@{}c@{}}0\\ (0.00)\end{tabular}} & \begin{tabular}[c]{@{}c@{}}0\\ (0.00)\end{tabular} \\ \cline{2-6} 
        \multicolumn{1}{|c|}{} & \textbf{坐姿} & \multicolumn{1}{c|}{\begin{tabular}[c]{@{}c@{}}0\\ (0.00)\end{tabular}} & \multicolumn{1}{c|}{\begin{tabular}[c]{@{}c@{}}3\\ (0.02)\end{tabular}} & \multicolumn{1}{c|}{{\color[HTML]{FE0000} \begin{tabular}[c]{@{}c@{}}196\\ (0.98)\end{tabular}}} & \begin{tabular}[c]{@{}c@{}}1\\ (0.01)\end{tabular} \\ \cline{2-6} 
        \multicolumn{1}{|c|}{\multirow{-4}{*}{\textbf{實際分類}}} & \multicolumn{1}{l|}{\textbf{站立}} & \multicolumn{1}{c|}{\begin{tabular}[c]{@{}c@{}}0\\ (0.00)\end{tabular}} & \multicolumn{1}{c|}{\begin{tabular}[c]{@{}c@{}}0\\ (0.00)\end{tabular}} & \multicolumn{1}{c|}{\begin{tabular}[c]{@{}c@{}}0\\ (0.00)\end{tabular}} & {\color[HTML]{FE0000} \begin{tabular}[c]{@{}c@{}}192\\ (1.00)\end{tabular}} \\ \hline
    \end{tabular}
\end{table}

\section{影片危險偵測實驗}
\subsection{實驗目的與設計}
本實驗為驗證此系統能基於嬰兒影像進行危險監測,
利用網路之真實嬰兒影片,
包含不同之拍攝視角、嬰兒樣貌及狀態等,
實驗臉部遮擋辨識模型與姿勢辨識模型之準確性。

\subsection{實驗評估方式}
本實驗透過輸出每幀影像之臉部遮擋及姿勢辨識結果,
計算其accuracy、precision及recall,
以驗證此二模型得以應用在監測嬰兒危險情境。

\subsection{實驗結果分析}
本實驗影片為嬰兒正躺於畫面中,
並包含使用奶嘴及未使用奶嘴之情境,
共切成3374幀影像,
將未拍攝到嬰兒畫面之影像刪除後,
剩餘3307張嬰兒影像進行辨識。

首先,
姿勢辨識的部分,
包含了278張誤判為趴躺姿勢的影像,
推測原因為嬰兒身體遭棉被遮擋(如\cref{fig:fig-error-video-posture}),
而只拍攝到露出的嬰兒臉部,
故造成姿勢辨識錯誤。
此部分之混淆矩陣如\cref{table:table-confusion-matrix-video-posture},
表中數字為各類影像之張數及占比。
\fig[0.5][fig:fig-error-video-posture][!hbt]{fig-error-video-posture.jpg}[姿勢辨識錯誤之影像:正躺誤判為趴躺]

\begin{table}[h]
    \centering
    \caption{實驗影片之姿勢辨識混淆矩陣}
    \label{table:table-confusion-matrix-video-posture}
    \begin{tabular}{|cc|cccc|}
        \hline
        \multicolumn{2}{|l|}{} & \multicolumn{4}{c|}{\textbf{預測分類}} \\ \cline{3-6} 
        \multicolumn{2}{|l|}{\multirow{-2}{*}{}} & \multicolumn{1}{c|}{\textbf{正躺}} & \multicolumn{1}{c|}{\textbf{趴躺}} & \multicolumn{1}{c|}{\textbf{坐姿}} & \multicolumn{1}{l|}{\textbf{站立}} \\ \hline
        \multicolumn{1}{|c|}{} & \textbf{正躺} & \multicolumn{1}{c|}{{\color[HTML]{FE0000} \begin{tabular}[c]{@{}c@{}}3029\\ (0.92)\end{tabular}}} & \multicolumn{1}{c|}{\begin{tabular}[c]{@{}c@{}}278\\ (0.08)\end{tabular}} & \multicolumn{1}{c|}{\begin{tabular}[c]{@{}c@{}}0\\ (0.00)\end{tabular}} & \begin{tabular}[c]{@{}c@{}}0\\ (0.00)\end{tabular} \\ \cline{2-6} 
        \multicolumn{1}{|c|}{} & \textbf{趴躺} & \multicolumn{1}{c|}{\begin{tabular}[c]{@{}c@{}}0\\ (0.00)\end{tabular}} & \multicolumn{1}{c|}{{\color[HTML]{FE0000} \begin{tabular}[c]{@{}c@{}}0\\ (0.00)\end{tabular}}} & \multicolumn{1}{c|}{\begin{tabular}[c]{@{}c@{}}0\\ (0.00)\end{tabular}} & \begin{tabular}[c]{@{}c@{}}0\\ (0.00)\end{tabular} \\ \cline{2-6} 
        \multicolumn{1}{|c|}{} & \textbf{坐姿} & \multicolumn{1}{c|}{\begin{tabular}[c]{@{}c@{}}0\\ (0.00)\end{tabular}} & \multicolumn{1}{c|}{\begin{tabular}[c]{@{}c@{}}0\\ (0.00)\end{tabular}} & \multicolumn{1}{c|}{{\color[HTML]{FE0000} \begin{tabular}[c]{@{}c@{}}0\\ (0.00)\end{tabular}}} & \begin{tabular}[c]{@{}c@{}}0\\ (0.00)\end{tabular} \\ \cline{2-6} 
        \multicolumn{1}{|c|}{\multirow{-4}{*}{\textbf{實際分類}}} & \multicolumn{1}{l|}{\textbf{站立}} & \multicolumn{1}{c|}{\begin{tabular}[c]{@{}c@{}}0\\ (0.00)\end{tabular}} & \multicolumn{1}{c|}{\begin{tabular}[c]{@{}c@{}}0\\ (0.00)\end{tabular}} & \multicolumn{1}{c|}{\begin{tabular}[c]{@{}c@{}}0\\ (0.00)\end{tabular}} & {\color[HTML]{FE0000} \begin{tabular}[c]{@{}c@{}}0\\ (0.00)\end{tabular}} \\ \hline
    \end{tabular}
\end{table}

其次,
臉部遮擋辨識的部分,
包含36張未拍攝至嬰兒臉部而未被偵測之影像。
而其中多張影像類別應為嬰兒正在使用奶嘴或安全狀態,
但誤判為遭異物遮蔽之警示狀態,
推測原因為影像中之奶嘴或嬰兒臉部遭手部等遮擋(如\cref{fig:fig-error-video-face}),
而誤判為類別,
此部分之混淆矩陣如\cref{table:table-confusion-matrix-video-face},
表中數字為各類影像之張數及占比。
\fig[0.4][fig:fig-error-video-face][!hbt]{fig-error-video-face.jpg}[臉部遮擋誤判之為警示狀態]

\begin{table}[h]
    \centering
    \caption{實驗影片之臉部遮擋辨識混淆矩陣}
    \label{table:table-confusion-matrix-video-face}
    \begin{tabular}{|cc|ccc|}
        \hline
        \multicolumn{2}{|l|}{} & \multicolumn{3}{c|}{\textbf{預測分類}} \\ \cline{3-5} 
        \multicolumn{2}{|l|}{\multirow{-2}{*}{}} & \multicolumn{1}{c|}{\textbf{無遮蔽}} & \multicolumn{1}{c|}{\textbf{使用奶嘴}} & \textbf{有異物} \\ \hline
        \multicolumn{1}{|c|}{} & \textbf{無遮蔽}  & \multicolumn{1}{c|}{{\color[HTML]{FE0000} \begin{tabular}[c]{@{}c@{}}1223\\ (0.84)\end{tabular}}} & \multicolumn{1}{c|}{\begin{tabular}[c]{@{}c@{}}52\\ (0.04)\end{tabular}} & \begin{tabular}[c]{@{}c@{}}184\\ (0.13)\end{tabular} \\ \cline{2-5} 
        \multicolumn{1}{|c|}{} & \textbf{使用奶嘴} & \multicolumn{1}{c|}{\begin{tabular}[c]{@{}c@{}}17\\ (0.01)\end{tabular}} & \multicolumn{1}{c|}{{\color[HTML]{FE0000} \begin{tabular}[c]{@{}c@{}}1308\\ (0.78)\end{tabular}}} & \begin{tabular}[c]{@{}c@{}}358\\ (0.21)\end{tabular} \\ \cline{2-5} 
        \multicolumn{1}{|c|}{\multirow{-3}{*}{\textbf{實際分類}}} & \textbf{有異物} & \multicolumn{1}{c|}{\begin{tabular}[c]{@{}c@{}}0\\ (0.00)\end{tabular}} & \multicolumn{1}{c|}{\begin{tabular}[c]{@{}c@{}}0\\ (0.00)\end{tabular}} & {\color[HTML]{FE0000} \begin{tabular}[c]{@{}c@{}}129\\ (1.00)\end{tabular}} \\ \hline
    \end{tabular}
\end{table}

\end{document}