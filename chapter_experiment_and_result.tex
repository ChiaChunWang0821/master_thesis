\documentclass[class=NCU_thesis, crop=false]{standalone}
\begin{document}

\chapter{實驗設計與結果}

\section{嬰兒臉部偵測實驗}
\subsection{實驗目的與設計}
在收集嬰兒臉部資料集時,
需針對嬰兒影像擷取出臉部範圍,
進而後續之臉部遮擋辨識階段。

因此,本實驗使用3.3節之嬰兒姿勢資料集,
就OpenCV~\cite{goyal_face_2017}
、SSD~\cite{ye_face_2021}
、MTCNN~\cite{xiang_joint_2017}
及RetinaFace~\cite{deng_retinaface_2020}
等臉部偵測演算法,
分析其執行時間及臉部擷取準確度進行比較,
以驗證適合本系統之演算法。

\subsection{實驗評估方式}
本實驗為驗證嬰兒臉部偵測演算法之實際可行性,
將針對臉部偵測執行時間及偵測結果之準確度分別進行比較:
透過計算演算法偵測所有資料集共15416張影像所花費之時間,
得以算出各演算法平均每張需花費之時間;
而準確度則將嬰兒臉部偵測之影像結果進行分類標註,
分別計算出各演算法之accuracy、precision及recall。

\subsection{實驗結果與分析}
首先,針對演算法之執行時間進行比較,
透過實驗結果可得出使用SSD演算法進行嬰兒臉部偵測,
將可擁有較佳的偵測速度。
而四項演算法偵測15416張影像之詳細實驗結果如下:

(1)OpenCV演算法:
共花費18分01.78秒,
平均每張影像需花0.07秒;

(2)SSD演算法:
共花費9分17.26秒,
平均每張影像需花0.04秒;

(3)MTCNN演算法:
共花費2小時8分22.05秒,
平均每張影像需花0.50秒;

(4)RetinaFace演算法:
共花費XX,
平均每張影像需花XX秒。

接著,就偵測之精確度進行比較,
透過實驗結果可得出選用RetinaFace演算法進行嬰兒臉部偵測,
可擁有較佳的偵測準確度。
而四項演算法進行嬰兒臉部偵測之詳細實驗結果如下:

(1)使用OpenCV演算法偵測結果如\cref{table:table-opencv},
由於偵測效果不佳,
將多數影像皆誤判為False(無臉),
故僅計算其precision為79.90\%;

(2)使用SSD演算法偵測結果如\cref{table:table-ssd},
由於偵測效果不佳,
將多數影像皆誤判為False(無臉),
故僅計算其precision為99.90\%;

(3)使用MTCNN演算法偵測結果如\cref{table:table-mtcnn},
其accuracy為90.20\%、precision為94.76\%以及recall為90.93\%;

(4)使用RetinaFace演算法偵測結果如\cref{table:table-retinaface},
其accuracy為99.78\%、precision為99.75\%以及recall為99.91\%。

綜觀上述兩部分實驗結果,
若系統欲擁有較迅速的執行速度又兼具偵測準確度,
可得出以下結論:
先使用SSD演算法找尋嬰兒臉部範圍,
雖然此方法在許多狀況未能如期找到嬰兒臉部範圍,
但其準確度很高,
故能利用此算法之時間優勢;
而若SSD演算法找不到嬰兒臉部時,
則接續使用RetinaFace演算法,
利用其很高之正確率及準確率之特質進行嬰兒臉部偵測。

\begin{table}[h]
    \centering
    \caption{OpenCV演算法偵測嬰兒臉部結果}
    \label{table:table-opencv}
    \begin{tabular}{ccc}
    \hline
     & True(預測有臉)& False(預測無臉)\\
    \hline
    True & \multirow{2}{*}{2882} & \multirow{4}{*}{11809} \\
    (實際有臉)& & \\
    False & \multirow{2}{*}{725} & \\
    (實際無臉)&  & \\
    \hline
    \end{tabular}
\end{table}

\begin{table}[h]
    \centering
    \caption{SSD演算法偵測嬰兒臉部結果}
    \label{table:table-ssd}
    \begin{tabular}{ccc}
    \hline
     & True(預測有臉)& False(預測無臉)\\
    \hline
    True & \multirow{2}{*}{4830} & \multirow{4}{*}{10581} \\
    (實際有臉)& & \\
    False & \multirow{2}{*}{5} & \\
    (實際無臉)&  & \\
    \hline
    \end{tabular}
\end{table}

\begin{table}[h]
    \centering
    \caption{MTCNN演算法偵測嬰兒臉部結果}
    \label{table:table-mtcnn}
    \begin{tabular}{ccc}
    \hline
     & True(預測有臉)& False(預測無臉)\\
    \hline
    True & \multirow{2}{*}{9361} & \multirow{2}{*}{994} \\
    (實際有臉)& & \\
    False & \multirow{2}{*}{517} & \multirow{2}{*}{4544} \\
    (實際無臉)&  & \\
    \hline
    Total & \textbf{9878} & \textbf{5538} \\
    \hline
    \end{tabular}
\end{table}

\begin{table}[h]
    \centering
    \caption{RetinaFace演算法偵測嬰兒臉部結果}
    \label{table:table-retinaface}
    \begin{tabular}{ccc}
    \hline
     & True(預測有臉)& False(預測無臉)\\
    \hline
    True & \multirow{2}{*}{12925} & \multirow{2}{*}{11} \\
    (實際有臉)& & \\
    False & \multirow{2}{*}{33} & \multirow{2}{*}{2447} \\
    (實際無臉)&  & \\
    \hline
    Total & \textbf{12958} & \textbf{2458} \\
    \hline
    \end{tabular}
\end{table}

\section{臉部遮擋辨識實驗}
\subsection{實驗目的與設計}
本系統為偵測嬰兒臉部是否遭非奶嘴之異物遮擋,
使用3.2節之資料集以ResNet50~\cite{he_deep_2016}
訓練模型,
並透過預測集進行模型驗證。

程式實作中,
網路訓練回合數為20,
設定影像資料大小為224x224,
包含三個類別(臉部無遮擋之安全狀態、臉部遭奶嘴遮擋及臉部遭異物遮擋之危險狀態),
且透過data augmentation技術生成訓練及測試資料,
輸出層使用softmax作為激發函數,
並使用Adam作為優化器且將學習率設為0.000001以進行微調。

\subsection{實驗結果分析}
本實驗訓練之模型其準確率達98\%,
詳細訓練結果請見\cref{fig:fig-result-face}。
\fig[0.8][fig:fig-result-face][!hbt]{fig-result-face.png}[臉部辨識訓練結果][臉部辨識訓練結果]

接著,再使用342張之預測集影像進行模型驗證,
所有影像皆辨識正確,
其混淆矩陣如\cref{fig:fig-confusion-matrix-face-occlusion}。
\fig[0.6][fig:fig-confusion-matrix-face-occlusion][!hbt]{fig-confusion-matrix-face-occlusion.png}[臉部遮擋辨識模型之混淆矩陣]

\section{姿勢分類實驗}
\subsection{模型訓練結果}
準確率達99\%,
詳細訓練結果請見\cref{fig:fig-result-four-classes}。
\fig[0.8][fig:fig-result-four-classes][!hbt]{fig-result-four-classes.png}[姿勢辨識訓練結果][姿勢辨識訓練結果]

\subsection{實驗設計}
使用ResNet50~\cite{he_deep_2016}
訓練臉部遮擋模型與奶嘴辨識模型,
訓練回合數皆為20。

\subsection{實驗評估方式}
姿勢分類實驗評估方式 姿勢分類實驗評估方式

\subsection{實驗結果分析}
我們使用744張影像針對此模型進行測試,
包含了五張類別辨識錯誤的影像,
其中有三張將坐姿誤判為趴躺姿勢,
推測原因為嬰兒雖呈現坐姿,
但上半身貼近其腿部,而導致誤判。

此模型之混淆矩陣,請見\cref{fig:fig-confusion-matrix-four-classes}。
\fig[1.1][fig:fig-confusion-matrix-four-classes][!hbt]{fig-confusion-matrix-four-classes.png}[嬰兒姿勢辨識之混淆矩陣][嬰兒姿勢辨識之混淆矩陣]

\section{影片危險偵測實驗}
\subsection{實驗設計}
影片危險偵測實驗 影片危險偵測實驗

\subsection{實驗評估方式}
影片危險偵測實驗 影片危險偵測實驗

\subsection{實驗結果分析}
影片危險偵測實驗 影片危險偵測實驗

\end{document}