\documentclass[class=NCU_thesis, crop=false]{standalone}
\begin{document}

\chapter{實驗設計與結果}

\section{嬰兒臉部偵測實驗}
\subsection{實驗目的與設計}
在收集嬰兒臉部資料集時,
需針對嬰兒影像擷取出臉部範圍,
進而後續之臉部遮擋辨識階段。

因此,本實驗使用3.3節之嬰兒姿勢資料集,
就OpenCV~\cite{goyal_face_2017}
、SSD~\cite{ye_face_2021}
、MTCNN~\cite{zhang_joint_2016}
及RetinaFace~\cite{deng_retinaface_2020}
等臉部偵測演算法,
分析其執行時間及臉部擷取準確度進行比較,
以驗證適合本系統之演算法。

\subsection{實驗評估方式}
本實驗為驗證嬰兒臉部偵測演算法之實際可行性,
將針對臉部偵測執行時間及偵測結果之準確度分別進行比較:
透過計算演算法偵測所有資料集共15416張影像所花費之時間,
得以算出各演算法平均每張需花費之時間;
而準確度則將嬰兒臉部偵測之影像結果進行分類標註,
分別計算出各演算法之accuracy、precision及recall。

\subsection{實驗結果與分析}
首先,針對演算法之執行時間進行比較,
透過實驗結果可得出使用SSD演算法進行嬰兒臉部偵測,
將可擁有較佳的偵測速度。
而四項演算法偵測15416張影像之詳細實驗結果如下:

(1)OpenCV演算法:
共花費18分01.78秒,
平均每張影像需花0.07秒;

(2)SSD演算法:
共花費9分17.26秒,
平均每張影像需花0.04秒;

(3)MTCNN演算法:
共花費2小時8分22.05秒,
平均每張影像需花0.50秒;

(4)RetinaFace演算法:
共花費5小時42分2.10秒,
平均每張影像需花1.33秒。

接著,就偵測之精確度進行比較,
透過實驗結果可得出選用RetinaFace演算法進行嬰兒臉部偵測,
可擁有較佳的偵測準確度。
而四項演算法進行嬰兒臉部偵測之詳細實驗結果如下:

(1)使用OpenCV演算法偵測結果如\cref{table:table-opencv},
由於偵測效果不佳,
將多數影像皆誤判為False(無臉),
故僅計算其precision為79.90\%;

(2)使用SSD演算法偵測結果如\cref{table:table-ssd},
由於偵測效果不佳,
將多數影像皆誤判為False(無臉),
故僅計算其precision為99.90\%;

(3)使用MTCNN演算法偵測結果如\cref{table:table-mtcnn},
其accuracy為90.20\%、precision為94.76\%以及recall為90.93\%;

(4)使用RetinaFace演算法偵測結果如\cref{table:table-retinaface},
其accuracy為99.78\%、precision為99.75\%以及recall為99.91\%。

綜觀上述兩部分實驗結果,
若系統欲擁有較迅速的執行速度又兼具偵測準確度,
可得出以下結論:
先使用SSD演算法找尋嬰兒臉部範圍,
雖然此方法在許多狀況未能如期找到嬰兒臉部範圍,
但其準確度很高,
故能利用此算法之時間優勢;
而若SSD演算法找不到嬰兒臉部時,
則接續使用RetinaFace演算法,
利用其很高之正確率及準確率之特質進行嬰兒臉部偵測。

\begin{table}[h]
    \centering
    \caption{OpenCV演算法偵測嬰兒臉部結果}
    \label{table:table-opencv}
    \begin{tabular}{ccc}
    \hline
     & True(預測有臉)& False(預測無臉)\\
    \hline
    True & \multirow{2}{*}{2882} & \multirow{4}{*}{11809} \\
    (實際有臉)& & \\
    False & \multirow{2}{*}{725} & \\
    (實際無臉)&  & \\
    \hline
    \end{tabular}
\end{table}

\begin{table}[h]
    \centering
    \caption{SSD演算法偵測嬰兒臉部結果}
    \label{table:table-ssd}
    \begin{tabular}{ccc}
    \hline
     & True(預測有臉)& False(預測無臉)\\
    \hline
    True & \multirow{2}{*}{4830} & \multirow{4}{*}{10581} \\
    (實際有臉)& & \\
    False & \multirow{2}{*}{5} & \\
    (實際無臉)&  & \\
    \hline
    \end{tabular}
\end{table}

\begin{table}[h]
    \centering
    \caption{MTCNN演算法偵測嬰兒臉部結果}
    \label{table:table-mtcnn}
    \begin{tabular}{ccc}
    \hline
     & True(預測有臉)& False(預測無臉)\\
    \hline
    True & \multirow{2}{*}{9361} & \multirow{2}{*}{994} \\
    (實際有臉)& & \\
    False & \multirow{2}{*}{517} & \multirow{2}{*}{4544} \\
    (實際無臉)&  & \\
    \hline
    Total & \textbf{9878} & \textbf{5538} \\
    \hline
    \end{tabular}
\end{table}

\begin{table}[h]
    \centering
    \caption{RetinaFace演算法偵測嬰兒臉部結果}
    \label{table:table-retinaface}
    \begin{tabular}{ccc}
    \hline
     & True(預測有臉)& False(預測無臉)\\
    \hline
    True & \multirow{2}{*}{12925} & \multirow{2}{*}{11} \\
    (實際有臉)& & \\
    False & \multirow{2}{*}{33} & \multirow{2}{*}{2447} \\
    (實際無臉)&  & \\
    \hline
    Total & \textbf{12958} & \textbf{2458} \\
    \hline
    \end{tabular}
\end{table}

\section{臉部遮擋辨識實驗}
\subsection{實驗目的與設計}
本系統為偵測嬰兒臉部是否遭非奶嘴之異物遮擋,
使用3.2節之資料集以ResNet50~\cite{he_deep_2016}
訓練模型,
並透過驗證集進行模型驗證。

程式實作中,
網路訓練回合數為20,
設定影像資料大小為224x224,
包含三個類別(臉部無遮擋之安全狀態、臉部遭奶嘴遮擋及臉部遭異物遮擋之危險狀態),
且透過data augmentation技術生成訓練及測試資料,
輸出層使用softmax作為激發函數,
並使用Adam作為優化器且將學習率設為0.000001以進行微調。

\subsection{實驗結果分析}
本實驗訓練之模型其最終訓練準確率達98.06\%,
而測試準確率達99.43\%,
詳細訓練結果請見\cref{fig:fig-result-face-occlusion}。
\fig[1][fig:fig-result-face-occlusion][!hbt]{fig-result-face-occlusion.png}[臉部辨識訓練結果][臉部辨識訓練結果]

接著,再使用342張之驗證集影像進行模型驗證,
所有影像皆辨識正確,
其混淆矩陣如\cref{fig:fig-confusion-matrix-face-occlusion}。
\fig[0.6][fig:fig-confusion-matrix-face-occlusion][!hbt]{fig-confusion-matrix-face-occlusion.png}[臉部遮擋辨識模型之混淆矩陣]

\section{姿勢辨識實驗}
\subsection{實驗目的與設計}
本系統為辨識嬰兒姿勢是否處於危險狀態,
使用3.3節之資料集以ResNet50~\cite{he_deep_2016}
訓練模型,
並透過驗證集進行模型驗證。

程式實作中,
網路訓練回合數為20,
設定影像資料大小為224x224,
包含四個類別(正躺、趴躺、坐姿及站立),
且透過data augmentation技術生成訓練及測試資料,
輸出層使用softmax作為激發函數,
並使用Adam作為優化器且將學習率設為0.000001以進行微調。

\subsection{實驗結果分析}
本實驗訓練之模型其最終訓練準確率達99.45\%,
而測試準確率達99.71\%,
詳細訓練結果請見\cref{fig:fig-result-four-classes}。
\fig[1][fig:fig-result-four-classes][!hbt]{fig-result-four-classes.png}[姿勢辨識訓練結果][姿勢辨識訓練結果]

接著,再使用744張之驗證集影像進行模型驗證,
包含了五張類別辨識錯誤的影像,
其混淆矩陣如\cref{fig:fig-confusion-matrix-four-classes}。
辨識錯誤之五張影像中,
有三張將坐姿誤判為趴躺姿勢,
推測原因為嬰兒雖呈現坐姿,
但上半身貼近其腿部(如\cref{fig:fig-error-four-classes}
),而導致誤判。
\fig[0.8][fig:fig-confusion-matrix-four-classes][!hbt]{fig-confusion-matrix-four-classes.png}[姿勢辨識模型之混淆矩陣]
\fig[0.6][fig:fig-error-four-classes][!hbt]{fig-error-four-classes.jpg}[姿勢辨識錯誤之影像:坐姿誤判為趴躺]

\section{影片危險偵測實驗}
\subsection{實驗目的與設計}
本實驗為驗證此系統能基於嬰兒影像進行危險監測,
利用網路之真實嬰兒影片,
包含不同之拍攝視角、嬰兒樣貌及狀態等,
實驗臉部遮擋辨識模型與姿勢辨識模型之準確性。

\subsection{實驗評估方式}
本實驗透過輸出每幀影像之臉部遮擋及姿勢辨識結果,
計算其accuracy、precision及recall,
以驗證此二模型得以應用在監測嬰兒危險情境。

\subsection{實驗結果分析}
本實驗影片為嬰兒正躺於畫面中,
並包含使用奶嘴及未使用奶嘴之情境,
共切成3374幀影像,
將未拍攝到嬰兒畫面之影像刪除後,
剩餘3307張嬰兒影像進行辨識。

首先,
姿勢辨識的部分,
包含了278張誤判為趴躺姿勢的影像,
推測原因為嬰兒身體遭棉被遮擋(如\cref{fig:fig-error-video-posture}),
而只拍攝到露出的嬰兒臉部,
故造成姿勢辨識錯誤,
其混淆矩陣如\cref{fig:fig-confusion-matrix-video-posture}。
\fig[0.6][fig:fig-error-video-posture][!hbt]{fig-error-video-posture.jpg}[姿勢辨識錯誤之影像:正躺誤判為趴躺]
\fig[0.8][fig:fig-confusion-matrix-video-posture][!hbt]{fig-confusion-matrix-video-posture.png}[實驗影片之姿勢辨識混淆矩陣]

其次,
臉部遮擋辨識的部分,
包含36張嬰兒臉部未被偵測之影像(如\cref{fig:fig-error-video-no-face}),
其中多張影像類別應為嬰兒正在使用奶嘴或安全狀態,
但誤判為遭異物遮蔽之警示狀態,
推測原因為影像中之奶嘴或嬰兒臉部遭手部等遮擋(如\cref{fig:fig-error-video-face}),
而誤判為類別,
其混淆矩陣如\cref{fig:fig-confusion-matrix-video-face}。
\fig[0.6][fig:fig-error-video-no-face][!hbt]{fig-error-video-no-face.jpg}[未偵測嬰兒臉部之影像]
\fig[0.4][fig:fig-error-video-face][!hbt]{fig-error-video-face.jpg}[臉部遮擋誤判之為警示狀態]
\fig[0.8][fig:fig-confusion-matrix-video-face][!hbt]{fig-confusion-matrix-video-face.png}[實驗影片之臉部遮擋辨識混淆矩陣]

\end{document}