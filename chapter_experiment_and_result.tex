\documentclass[class=NCU_thesis, crop=false]{standalone}
\usepackage{multirow}

\begin{document}

\chapter{實驗設計與結果}
本章節中,
根據第三章之研究方法以各項實驗驗證系統之設計,
並介紹各實驗之目的與設計、評估方式以及結果與分析,
透過以下五章子節進行說明:
臉部偵測準確度實驗、臉部偵測執行時間實驗、臉部遮擋辨識實驗、姿勢辨識實驗及影片危險偵測實驗。

另外,
本章各項實驗皆於相同硬體環境操作,詳細資訊如下:
\begin{itemize}
    \item 作業系統:Windows 10
    \item CPU:Intel(R) Core(TM) i7-10700KF CPU @ 3.80GHz
    \item 記憶體:128GB
    \item GPU:NVIDIA GeForce GTX 1660
\end{itemize}

\section{臉部偵測準確度實驗}
\label{sec:chapter_experiment_face_accuracy}
在收集嬰兒臉部資料集時,
需針對嬰兒影像擷取出臉部範圍,
進而後續之臉部遮擋辨識階段。

為了使本系統擁有較佳的臉部偵測準確性且兼具執行效能,
本文透過\ref{sec:chapter_experiment_face_accuracy}節及\ref{sec:chapter_experiment_face_time}節之實驗,
分別進行臉部偵測演算法準確度與執行時間之比較,
進而驗證以下設計:
先使用SSD演算法偵測嬰兒臉部,
此方法之召回率雖低,
但其準確度很高,
故能利用此算法之時間優勢;
而若SSD演算法找不到嬰兒面部時,
則接續使用RetinaFace演算法,
利用其正確率及準確率皆高之優點進行嬰兒臉部偵測。

\subsection{實驗目的與設計}
本實驗為計算人臉偵測演算法之嬰兒面部擷取準確度,
使用\ref{sec:chapter_method_posture_dataset}節的嬰兒姿勢資料集共15416張影像,
分析RetinaFace~\cite{deng_retinaface_2020}、MTCNN~\cite{zhang_joint_2016}、
SSD~\cite{ye_face_2021}及Haar cascade~\cite{goyal_face_2017}等演算法之偵測結果。

\subsection{實驗評估方式}
透過分類標註四項演算法偵測嬰兒臉部之結果影像,
計算出各演算法的accuracy、precision及recall。

\subsection{實驗結果與分析}
由\cref{table:table-retinaface}、\cref{table:table-mtcnn}、\cref{table:table-ssd}及\cref{table:table-opencv}
分別為RetinaFace、MTCNN、SSD及Haar cascade之詳細實驗結果。

\begin{table}[h]
    \centering
    \caption{RetinaFace~\cite{deng_retinaface_2020}偵測嬰兒臉部結果}
    \label{table:table-retinaface}
    \begin{tabular}{ccc}
    \hline
     & True(預測有臉)& False(預測無臉)\\
    \hline
    True & \multirow{2}{*}{12925} & \multirow{2}{*}{11} \\
    (實際有臉)& & \\
    False & \multirow{2}{*}{33} & \multirow{2}{*}{2447} \\
    (實際無臉)&  & \\
    \hline
    Total & \textbf{12958} & \textbf{2458} \\
    \hline
    \end{tabular}
\end{table}

\begin{table}[h]
    \centering
    \caption{MTCNN~\cite{zhang_joint_2016}偵測嬰兒臉部結果}
    \label{table:table-mtcnn}
    \begin{tabular}{ccc}
    \hline
     & True(預測有臉)& False(預測無臉)\\
    \hline
    True & \multirow{2}{*}{9361} & \multirow{2}{*}{3399} \\
    (實際有臉)& & \\
    False & \multirow{2}{*}{517} & \multirow{2}{*}{2140} \\
    (實際無臉)&  & \\
    \hline
    Total & \textbf{9877} & \textbf{5539} \\
    \hline
    \end{tabular}
\end{table}

\begin{table}[h]
    \centering
    \caption{SSD~\cite{ye_face_2021}偵測嬰兒臉部結果}
    \label{table:table-ssd}
    \begin{tabular}{ccc}
    \hline
     & True(預測有臉)& False(預測無臉)\\
    \hline
    True & \multirow{2}{*}{4830} & \multirow{2}{*}{8141} \\
    (實際有臉)& & \\
    False & \multirow{2}{*}{5} & \multirow{2}{*}{2440} \\
    (實際無臉)&  & \\
    \hline
    Total & \textbf{4835} & \textbf{10581} \\
    \hline
    \end{tabular}
\end{table}

\begin{table}[h]
    \centering
    \caption{Haar cascade~\cite{goyal_face_2017}偵測嬰兒臉部結果}
    \label{table:table-opencv}
    \begin{tabular}{ccc}
    \hline
     & True(預測有臉)& False(預測無臉)\\
    \hline
    True & \multirow{2}{*}{2882} & \multirow{2}{*}{9546} \\
    (實際有臉)& & \\
    False & \multirow{2}{*}{725} & \multirow{2}{*}{2263} \\
    (實際無臉)&  & \\
    \hline
    Total & \textbf{3607} & \textbf{11809} \\
    \hline
    \end{tabular}
\end{table}

而再經計算後,
四項演算法之accuracy、precision及recall值如\cref{table:table-face-detection-accuracy}。

\begin{table}[h]
    \centering
    \caption{人臉偵測準確度結果}
    \label{table:table-face-detection-accuracy}
    \begin{tabular}{cccc}
    \hline
    演算法 & Accuracy & Precision & Recall \\
    \hline
    RetinaFace & 99.71\% & 99.75\% & 99.91\% \\
    MTCNN & 74.60\% & 94.78\% & 73.36\% \\
    SSD & 47.16\% & 99.90\% & 37.24\% \\
    Haar cascade & 33.37\% & 79.90\% & 23.19\% \\
    \hline
    \end{tabular}
\end{table}

故透過本實驗結果可得出選用RetinaFace演算法行嬰兒臉部偵測,
可擁有較佳的偵測準確度。
另外,
值得注意的是,
雖然SSD的accuracy及recall很低,
但其precision達99.90\%,
也就是說其將多數影像誤判為無臉,
但判斷為有臉的結果幾乎正確。

因此,
本研究接續進行\ref{sec:chapter_experiment_face_time}節的實驗,
希望利用SSD這樣的特質。

\section{臉部偵測執行時間實驗}
\label{sec:chapter_experiment_face_time}
本研究進行嬰兒臉部偵測除了考量準確度外,
亦希望提升整體系統之執行效率。

\subsection{實驗目的與設計}
本實驗為計算人臉偵測演算法之執行時間,
使用\ref{sec:chapter_method_posture_dataset}節的嬰兒姿勢資料集共15416張影像,
分析RetinaFace~\cite{deng_retinaface_2020}、MTCNN~\cite{zhang_joint_2016}、
SSD~\cite{ye_face_2021}及Haar cascade~\cite{goyal_face_2017}等演算法,
其臉部偵測時間之結果;
再以嬰兒影片進行實驗,
比較本文所設計的兩步驟流程及僅使用RetinaFace演算法之偵測時間。

\subsection{實驗評估方式}
透過計算四項演算法偵測完整資料集所花費之時間,
分析各演算法平均偵測一張影像之執行時間;
並比較有無使用本文方法之臉部偵測時間。

\subsection{實驗結果與分析}
RetinaFace、MTCNN、SSD及Haar cascade四項演算法之詳細實驗結果,
請見\cref{table:table-face-detection-time}。
\begin{table}[h]
    \centering
    \caption{人臉偵測執行時間結果}
    \label{table:table-face-detection-time}
    \begin{tabular}{ccc}
    \hline
    演算法 & 總花費時間(15416張影像) & 每張影像平均時間 \\
    \hline
    RetinaFace & 5小時42分2.10秒 & 1.33秒 \\
    MTCNN & 2小時8分22.05秒 & 0.50秒 \\
    SSD & 9分17.26秒 & 0.04秒 \\
    Haar cascade & 18分01.78秒 & 0.07秒 \\
    \hline
    \end{tabular}
\end{table}

故透過本實驗結果可得出使用SSD演算法進行嬰兒臉部偵測,
將可擁有較佳的偵測速度。
而\ref{sec:chapter_experiment_face_accuracy}節實驗中,
準確度最高的RetinaFace其平均偵測一張影像需1.33秒,
為SSD的33.25倍。

而使用本文設計之兩步驟偵測流程及僅使用RetinaFace演算法之詳細實驗結果,
請見\cref{table:table-face-detection-time-2}。
\begin{table}[h]
    \centering
    \caption{使用本文方法之人臉偵測執行時間結果}
    \label{table:table-face-detection-time-2}
    \begin{tabular}{ccc}
    \hline
    方法 & 總花費時間(1594張影像) & 每張影像平均時間 \\
    \hline
    本文方法 & 17分51.21秒 & 0.67秒 \\
    僅使用RetinaFace & 32分2.69秒 & 1.21秒 \\
    \hline
    \end{tabular}
\end{table}

因此,
總結\ref{sec:chapter_experiment_face_accuracy}節與\ref{sec:chapter_experiment_face_time}節之實驗結果,
驗證本系統先使用SSD演算法偵測嬰兒臉部,
未如期找到目標時,
則改以RetinaFace演算法偵測,
達成兼具準確性及執行效率之系統目標。

\section{臉部遮擋辨識實驗}
本研究中,
利用深度學習技術辨識嬰兒臉部是否遭非奶嘴之異物遮蔽,
進而判斷嬰兒是否處於危險情境中。

\subsection{實驗目的與設計}
本實驗為訓練針對嬰兒臉部遮擋辨識之模型,
以ResNet50~\cite{he_deep_2016}訓練\ref{sec:chapter_method_face_dataset}節的嬰兒臉部資料集,
並透過驗證集進行模型驗證。

程式實作中,
網路訓練回合數為20,
設定影像資料大小為224x224,
包含三個類別(臉部無遮擋之安全狀態、使用奶嘴及面部遭異物遮蔽之警示狀態),
且透過data augmentation技術生成訓練及測試資料,
輸出層使用softmax作為激發函數,
並使用Adam作為optimizer且將學習率設為0.000001以進行微調。

\subsection{實驗結果分析}
模型最終訓練準確率達98.06\%,
而測試準確率達99.43\%,
詳細訓練結果請見\cref{fig:fig-result-face-occlusion}。
\fig[0.9][fig:fig-result-face-occlusion][!hbt]{fig-result-face-occlusion.png}[臉部辨識訓練及測試結果][臉部辨識訓練及測試結果]

我們使用342張之驗證集影像進行模型驗證,
所有影像皆辨識正確。
此模型之混淆矩陣如\cref{table:table-confusion-matrix-face-occlusion},
表中數字為各類影像之張數及比例,
並於最右行展示各類別召回率。
\begin{table}[h]
    \centering
    \caption{臉部遮擋辨識模型之混淆矩陣(單位:張 / 百分比)}
    \label{table:table-confusion-matrix-face-occlusion}
    \begin{tabular}{|cc|ccc|c|}
        \hline
        \multicolumn{2}{|l|}{}                                                                                              & \multicolumn{3}{c|}{\textbf{預測類別}}                                                                                                                    & \multicolumn{1}{c|}{\textbf{召回率}} \\ \cline{3-5} 
        \multicolumn{2}{|l|}{\multirow{-2}{*}{}}                                                                            & \multicolumn{1}{c|}{\textbf{安全}}                       & \multicolumn{1}{c|}{\textbf{奶嘴}}                       & \textbf{警示}                       & \multicolumn{1}{c|}{}                \\ \hline
        \multicolumn{1}{|c|}{}                                                                              & \textbf{安全} & \multicolumn{1}{c|}{{\color[HTML]{FE0000} 120 (100.00)}} & \multicolumn{1}{c|}{0 (0.00)}                            & 0 (0.00)                            & 100.00\%                             \\ \cline{2-6} 
        \multicolumn{1}{|c|}{}                                                                              & \textbf{奶嘴} & \multicolumn{1}{c|}{0 (0.00)}                            & \multicolumn{1}{c|}{{\color[HTML]{FE0000} 115 (100.00)}} & 0 (0.00)                            & 100.00\%                             \\ \cline{2-6} 
        \multicolumn{1}{|c|}{\multirow{-3}{*}{\textbf{\begin{tabular}[c]{@{}c@{}}實際\\類別\end{tabular}}}} & \textbf{警示} & \multicolumn{1}{c|}{0 (0.00)}                            & \multicolumn{1}{c|}{0 (0.00)}                            & {\color[HTML]{FE0000} 107 (100.00)} & 100.00\%                             \\ \hline
    \end{tabular}
\end{table}

\section{姿勢辨識實驗}
本研究中,
利用深度學習技術辨識嬰兒基礎姿勢,
進而判斷嬰兒是否處於危險情境中。

\subsection{實驗目的與設計}
本實驗為訓練針對嬰兒姿勢辨識之模型,
以ResNet50~\cite{he_deep_2016}訓練\ref{sec:chapter_method_posture_dataset}節的嬰兒姿勢資料集,
並透過驗證集進行模型驗證。

程式實作中,
網路訓練回合數為20,
設定影像資料大小為224x224,
包含四個類別(正躺、趴躺、坐姿及站立),
且透過data augmentation技術生成訓練及測試資料,
輸出層使用softmax作為激發函數,
並使用Adam作為optimizer且將學習率設為0.000001以進行微調。

\subsection{實驗結果分析}
模型最終訓練準確率達99.45\%,
而測試準確率達99.71\%,
詳細訓練結果請見\cref{fig:fig-result-four-classes}。
\fig[0.9][fig:fig-result-four-classes][!hbt]{fig-result-four-classes.png}[姿勢辨識訓練及測試結果][姿勢辨識訓練及測試結果]

我們使用744張之驗證集進行模型驗證,
包含了五張類別辨識錯誤的影像,
其中三張將坐姿誤判為趴躺姿勢,
推測原因為嬰兒雖呈現坐姿,
但上半身貼近其腿部(如\cref{fig:fig-error-four-classes}),
而導致誤判。
此模型之混淆矩陣如\cref{table:table-confusion-matrix-four-classes},
表中數字為各類影像之張數及比例,
並於最右行展示各類別召回率。
\fig[0.55][fig:fig-error-four-classes][!hbt]{fig-error-four-classes.jpg}[姿勢辨識錯誤之影像:坐姿誤判為趴躺]
\begin{table}[h]
    \centering
    \caption{姿勢辨識模型之混淆矩陣(單位:張 / 百分比)}
    \label{table:table-confusion-matrix-four-classes}
    \begin{tabular}{|cc|cccc|c|}
        \hline
        \multicolumn{2}{|c|}{}                                                                                              & \multicolumn{4}{c|}{\textbf{預測類別}}                                                                                                                                                                             & \multicolumn{1}{c|}{\textbf{召回率}} \\ \cline{3-6} 
        \multicolumn{2}{|c|}{\multirow{-2}{*}{}}                                                                            & \multicolumn{1}{c|}{\textbf{正躺}}                       & \multicolumn{1}{c|}{\textbf{趴躺}}                      & \multicolumn{1}{c|}{\textbf{坐姿}}                      & \textbf{站立}                       & \multicolumn{1}{c|}{}                \\ \hline
        \multicolumn{1}{|c|}{}                                                                              & \textbf{正躺} & \multicolumn{1}{c|}{{\color[HTML]{FE0000} 164 (100.00)}} & \multicolumn{1}{c|}{0 (0.00)}                           & \multicolumn{1}{c|}{0 (0.00)}                           & 0 (0.00)                            & 100.00\%                             \\ \cline{2-7} 
        \multicolumn{1}{|c|}{}                                                                              & \textbf{趴躺} & \multicolumn{1}{c|}{1 (0.52)}                            & \multicolumn{1}{c|}{{\color[HTML]{FE0000} 191 (99.48)}} & \multicolumn{1}{c|}{0 (0.00)}                           & 0 (0.00)                            & 99.48\%                              \\ \cline{2-7} 
        \multicolumn{1}{|c|}{}                                                                              & \textbf{坐姿} & \multicolumn{1}{c|}{0 (0.00)}                            & \multicolumn{1}{c|}{3 (1.50)}                           & \multicolumn{1}{c|}{{\color[HTML]{FE0000} 196 (98.00)}} & 1 (0.50)                            & 98.00\%                              \\ \cline{2-7} 
        \multicolumn{1}{|c|}{\multirow{-4}{*}{\textbf{\begin{tabular}[c]{@{}c@{}}實際\\類別\end{tabular}}}} & \textbf{站立} & \multicolumn{1}{c|}{0 (0.00)}                            & \multicolumn{1}{c|}{0 (0.00)}                           & \multicolumn{1}{c|}{0 (0.00)}                           & {\color[HTML]{FE0000} 192 (100.00)} & 100.00\%                             \\ \hline
    \end{tabular}
\end{table}

\section{影片危險偵測實驗}
本研究基於嬰兒影像進行臉部遮擋及姿勢辨識,
透過讀取嬰兒影片達成危險監測之目標。

\subsection{實驗目的與設計}
本實驗為驗證此系統能基於嬰兒影像進行危險監測,
利用網路之真實嬰兒影片,
包含不同之拍攝視角、嬰兒樣貌及狀態等,
實驗臉部遮擋辨識模型與姿勢辨識模型之準確性。

\subsection{實驗評估方式}
透過輸出每幀影像之臉部遮擋及姿勢辨識結果,
計算其accuracy、precision及recall,
以驗證此二模型得以應用在監測嬰兒危險情境。

\subsection{實驗結果分析}
本實驗使用之影片,
包含嬰兒許多的不同情境,
如:清醒與否、不同衣著、穿戴帽子與否及相異背景環境等,
進行其臉部遮擋(無遮蔽、使用奶嘴及遭異物遮擋)與姿勢(正躺、趴躺、坐姿及站立)之危險辨識。

首先,
姿勢辨識的部分,
包含了多張誤判為趴躺姿勢的影像,
推測原因為嬰兒身體遭棉被遮擋(如\cref{fig:fig-error-video-posture}),
而只拍攝到露出的嬰兒臉部,
故造成姿勢辨識錯誤。
\fig[0.55][fig:fig-error-video-posture][!hbt]{fig-error-video-posture.jpg}[姿勢辨識錯誤之影像:正躺誤判為趴躺]

其次,
臉部遮擋辨識的部分,
會先刪去嬰兒臉部未被偵測之影像(如\cref{fig:fig-error-video-no-face}),
而後判斷有多張影像類別應為嬰兒正在使用奶嘴或安全狀態,
但誤判為遭異物遮蔽之警示狀態,
推測原因為影像中之奶嘴或嬰兒臉部遭手部等遮擋(如\cref{fig:fig-error-video-face}),
而誤判類別。
\fig[0.55][fig:fig-error-video-no-face][!hbt]{fig-error-video-no-face.jpg}[未偵測嬰兒臉部之影像]
\fig[0.4][fig:fig-error-video-face][!hbt]{fig-error-video-face.jpg}[臉部遮擋誤判之為警示狀態]

兩部分之混淆矩陣如\cref{table:table-confusion-matrix-video-posture}及\cref{table:table-confusion-matrix-video-face},
表中數字為各類影像之張數及比例,
並於最右行展示各類別召回率。
\begin{table}[h]
    \centering
    \caption{實驗影片姿勢辨識之混淆矩陣(單位:張 / 百分比)}
    \label{table:table-confusion-matrix-video-posture}
    \begin{tabular}{|cc|cccc|c|}
        \hline
        \multicolumn{2}{|c|}{}                                                                                              & \multicolumn{4}{c|}{\textbf{預測類別}}                                                                                                                                                                              & \multicolumn{1}{c|}{\textbf{召回率}} \\ \cline{3-6} 
        \multicolumn{2}{|c|}{\multirow{-2}{*}{}}                                                                            & \multicolumn{1}{c|}{\textbf{正躺}}                       & \multicolumn{1}{c|}{\textbf{趴躺}}                       & \multicolumn{1}{c|}{\textbf{坐姿}}                       & \textbf{站立}                      & \multicolumn{1}{c|}{}                \\ \hline
        \multicolumn{1}{|c|}{}                                                                              & \textbf{正躺} & \multicolumn{1}{c|}{{\color[HTML]{FE0000} 3223 (90.87)}} & \multicolumn{1}{c|}{324 (9.13)}                          & \multicolumn{1}{c|}{0 (0.00)}                            & 0 (0.00)                           & 90.87\%                              \\ \cline{2-7} 
        \multicolumn{1}{|c|}{}                                                                              & \textbf{趴躺} & \multicolumn{1}{c|}{0 (0.00)}                            & \multicolumn{1}{c|}{{\color[HTML]{FE0000} 161 (100.00)}} & \multicolumn{1}{c|}{0 (0.00)}                            & 0 (0.00)                           & 100.00\%                             \\ \cline{2-7} 
        \multicolumn{1}{|c|}{}                                                                              & \textbf{坐姿} & \multicolumn{1}{c|}{0 (0.00)}                            & \multicolumn{1}{c|}{0 (0.00)}                            & \multicolumn{1}{c|}{{\color[HTML]{FE0000} 214 (100.00)}} & 0 (0.00)                           & 100.00\%                             \\ \cline{2-7} 
        \multicolumn{1}{|c|}{\multirow{-4}{*}{\textbf{\begin{tabular}[c]{@{}c@{}}實際\\類別\end{tabular}}}} & \textbf{站立} & \multicolumn{1}{c|}{4 (1.87)}                            & \multicolumn{1}{c|}{1 (0.47)}                            & \multicolumn{1}{c|}{0 (0.00)}                            & {\color[HTML]{FE0000} 209 (97.66)} & 97.66\%                              \\ \hline
    \end{tabular}
\end{table}

\begin{table}[h]
    \centering
    \caption{實驗影片臉部遮擋辨識之混淆矩陣(單位:張 / 百分比)}
    \label{table:table-confusion-matrix-video-face}
    \begin{tabular}{|cc|ccc|c|}
        \hline
        \multicolumn{2}{|c|}{}                                                                                              & \multicolumn{3}{c|}{\textbf{預測類別}}                                                                                                                    & \multicolumn{1}{c|}{\textbf{召回率}} \\ \cline{3-5} 
        \multicolumn{2}{|c|}{\multirow{-2}{*}{}}                                                                            & \multicolumn{1}{c|}{\textbf{安全}}                       & \multicolumn{1}{c|}{\textbf{奶嘴}}                       & \textbf{警示}                       & \multicolumn{1}{c|}{}                \\ \hline
        \multicolumn{1}{|c|}{}                                                                              & \textbf{安全} & \multicolumn{1}{c|}{{\color[HTML]{FE0000} 1429 (85.83)}} & \multicolumn{1}{c|}{52 (3.12)}                           & 184 (11.05)                         & 85.83\%                              \\ \cline{2-6} 
        \multicolumn{1}{|c|}{}                                                                              & \textbf{奶嘴} & \multicolumn{1}{c|}{17 (0.84)}                           & \multicolumn{1}{c|}{{\color[HTML]{FE0000} 1615 (80.23)}} & 381 (18.93)                         & 80.23\%                              \\ \cline{2-6} 
        \multicolumn{1}{|c|}{\multirow{-3}{*}{\textbf{\begin{tabular}[c]{@{}c@{}}實際\\類別\end{tabular}}}} & \textbf{警示} & \multicolumn{1}{c|}{40 (11.90)}                          & \multicolumn{1}{c|}{0 (0.00)}                            & {\color[HTML]{FE0000} 296 (88.10)}  & 88.10\%                              \\ \hline
    \end{tabular}
\end{table}

\end{document}