\documentclass[class=NCU_thesis, crop=false]{standalone}
\begin{document}

\chapter{相關研究}

\section{嬰兒猝死症}
嬰兒猝死症(The Sudden Infant Death Syndrome, 簡稱SIDS)
~\cite{kinney_sudden_2009}
之特徵為一位看似健康的嬰兒在睡眠期間突然死亡,
其真正致死之原因尚不明確且非單一。

目前醫界雖未有單一定義此症之直接致死原因,
但可統整出多項促使嬰兒猝死症發生之風險因素,
可分為兩類因素:
其一為外在因素,包含嬰兒因俯臥、側睡或蓋住面部等致使呼吸困難;
其二為內在因素,包含發展因素(如:早產)、
遺傳因素(如:家族性之嬰兒猝死症)、
性別(男性比例為女性的兩倍)或種族等。
除此之外,嬰兒也可能因其他外在環境條件,
如:產前或產後暴露於不良物質中(香菸煙霧、酒精或非法藥物等),
而弱化嬰兒之內在條件。

在嬰兒猝死症研究中,
有許多關於此症之死亡機制理論,
其中心肺控制假說主導了多數研究,
也造就了往後關於嬰兒猝死症之研究多基於嬰兒呼吸或自主神經機制的缺陷。
這樣的論點主要包含了五個步驟:
(1)發生危及生命的事件(如:面部朝下或面部遭遮蔽,將造成反射性或阻塞性呼吸暫停),
而將導致嬰兒窒息、腦部灌注不足或兩者皆發生。
(2)嬰兒無法自行轉頭,以應付窒息的情境,而導致無法從呼吸暫停中恢復。
(3)持續的窒息導致失去意識或反射,即低氧昏迷。
(4)發生心率過緩及缺氧喘氣,此現象在嬰兒因嬰兒猝死症逝世前將明顯發生。
(5)嬰兒之自主復甦能力受損,即因無效的喘氣而最終導致呼吸暫停及死亡。
因此,由嬰兒猝死症之紀錄中,
可看出此症狀並非一種突發疾病,
而是在嬰兒死亡前,
即會出現心率不正常或呼吸暫停之惡性循環現象。

另外,
醫界亦發現俯臥睡姿將會使嬰兒猝死症之風險增加三倍以上,
故在1990年代初期國際間即提倡嬰兒仰臥睡姿,
嬰兒猝死症之發病率也因此降低了50\% 以上,
但仍為嬰兒主要死亡原因之一。

\section{嬰兒監測系統}
在照護嬰兒的過程中,
由於嬰兒尚未發展出語言能力表達自己的不適,
或尚無能力將自己避免於危險之外。
因此,為了協助照顧者關注嬰兒狀態,
現有許多為自動化監測嬰兒之研究,
主要分為以感測器偵測生理訊號及以影像式偵測兩種方式。
% 兩種方式的優缺點比較

\subsection{感測器偵測}
此種方式利用多種不同感測器進行生理訊號之偵測,
包含利用呼吸感測器、濕度感測器、溫度感測器、非接觸式紅外溫度感測器、三軸加速度計、慣性感測器、一氧化碳感測器、二氧化碳感測器等,
分別量測嬰兒之呼吸頻率、出汗狀況、體溫、心率、身體位置或方向、睡眠姿勢、嬰兒周圍的一氧化碳濃度、呼出的二氧化碳濃度的變化等,
且多會透過物聯網技術開發出可穿戴式裝置之系統。

如:Linti等人~\cite{linti_sensory_2006}
所開發的嬰兒感測背心,
其將多個感官元件融入紡織品中以用來量測嬰兒之呼吸、心率、溫度及濕度;
Ferreira等人~\cite{ferreira_smart_2016}
開發了將感測器裝設於胸帶中,
而得以量測嬰兒之體溫、心率、呼吸頻率及身體位置,
並透過ZigBee技術將收集到的數據傳送至伺服器,
用戶則可透過醫療網頁介面進行查看及收到緊急訊息;
Ziganshin等人~\cite{ziganshin_uwb_2010}
基於超寬頻技術開發出可監測嬰兒呼吸及心率之系統,
其可檢測嬰兒之睡眠、清醒及警示狀態;
Lin等人~\cite{lin_wireless_2014}
開發出一套在嬰兒胸帶上嵌入了三種不同感測器的系統,
透過三軸加速度計可確定嬰兒睡姿(面朝上、下、左或右)與計算z軸資訊得出呼吸頻率、
利用溫度感測器量測體溫以及使用一氧化碳感測器偵測嬰兒周圍之一氧化碳濃度,
再藉由WiFi模組傳送收集之生理資料至伺服器,
而其驗證所計算之呼吸頻率準確率達100\%。

此種利用感測器監測嬰兒的方法,
雖然可直接量測嬰兒之生理訊號以判斷狀態正常與否,
但仍可能因硬體設備之缺陷無法準確量測,
進而有失判斷準確性,
亦或者因嬰兒需額外穿戴裝置而造成不適,
進而影響嬰兒活動或導致更多危險的發生。

\subsection{影像式偵測}
此種方式利用電腦視覺技術對於嬰兒影像畫面進行偵測,
現有研究中包含了計算嬰兒之呼吸頻率、關注於嬰兒之面部特徵及嬰兒趴睡姿勢偵測。

Fang等人~\cite{fang_vision-based_2015}
開發了一基於視覺之非接觸式呼吸頻率偵測系統,
先判斷嬰兒是否正在運動(包含頭部、四肢及身體運動,但不包含因呼吸引起的輕微運動),
若未偵測到嬰兒運動,
則系統開始進行呼吸頻率偵測:
首先,透過空間特徵擷取呼吸之候選點;
接著,利用模糊積分技術選擇呼吸點;
最終,得以計算嬰兒的呼吸頻率,進而可判斷嬰兒是否發生呼吸異常之情形。

Liu等人~\cite{liu_video-based_2017}
利用夜視攝影機拍攝在嬰兒床內的嬰兒,
並使用MIT所提出之Eulerian Magnification技術,
放大影片中的細微運動以監測拍攝對象之胸部運動,
若經正規化之像素差異值低於設定閥值,
則判斷其呼吸頻率異常,
進而透過手機裝置發出警報。

Gallo等人~\cite{gallo_marrsids_2019}
提出一名為MARRSIDS的模型,
其利用OpenCV之Haar-Like Features偵測嬰兒之面部特徵。
系統透過嬰兒臉部辨識與否及睜眼狀態,
判斷其是否處於危險情境中,
而需發出聲音警示:
若臉部未被偵測,
則認為嬰兒可能位於不良姿勢需發出警示;
而若嬰兒為睜眼狀態,
則代表嬰兒處於清醒狀態,
並非處於風險中。

Wang等人~\cite{wang_multi-task_2019}
提出了一個多任務貝氏深度神經架構,
其使用MobileNetV2網路,
針對自行收集之YunInfants資料集進行嬰兒頭部影像分析,
包含了四項子任務以達成嬰兒面部遮擋之監測:
(1)眼睛、鼻子或嘴巴是否可見,
(2)不可見的原因是否為被外物(如:枕頭)遮擋,
(3)眼睛睜開與否,
及(4)五個臉部座標之位置。

Bharati等人~\cite{bharati_efficient_2021}
提出一個基於卷積神經網絡的電腦視覺系統,
可用來評估嬰兒三種睡眠姿勢:
仰臥(正常狀態)、從仰臥轉換到趴臥(警示狀態)、趴臥(危險狀態),
並於嬰兒呈現趴臥姿勢時,
透過手機提醒照護人員。
而此系統亦提供了驗證反饋機制,
供照護人員對於系統警報是否誤報之回饋。
另外,由於目前未有公開之嬰兒姿勢資料集,
此文透過拍攝和真實嬰兒相同比例之娃娃進行資料收集。

現有研究中,
多關注於嬰兒呼吸運動、面部特徵或單一姿勢偵測,
而尚未有對於嬰兒常見動作之辨識模型,
故我們提出一可偵測嬰兒基礎姿勢及面部遮擋之危險監測系統。

\section{ResNet}
過往神經網路訓練中,
更深層的網路會有模型退化的問題,
亦即隨著網路深度的增加,準確率達飽和後,反而迅速下降,
而這樣的結果並非因過度擬合所致,
如\cref{fig:fig-training-error}
\fig[0.9][fig:fig-training-error][!hbt]{fig-plain-network.png}[網路深度與訓練誤差關係][網路深度與訓練誤差關係]
可看到兩個不同層數的網路其訓練誤差值。

因此,He等人~\cite{he_deep_2016}
提出了一個深度殘差學習(
\cref{fig:fig-residual-learning}
\fig[0.6][fig:fig-residual-learning][!hbt]{fig-residual-learning.png}[殘差學習][殘差學習]
)的架構,
利用shortcut connection執行identity mapping,
如此並不需要增加額外的參數,
亦即不增加計算複雜度。最終,本研究以152層的殘差網路在ILSVRC 2015中獲得第一名,
此網路比VGG網路深八倍,
卻仍擁有較低的複雜度。

\section{RetinaFace}
% \subsection{MTCNN}
% MTCNN MTCNN
% ~\cite{xiang_joint_2017}

% \subsection{RetinaFace}
RetinaFace
~\cite{deng_retinaface_2020}
是由Deng等人於2020年提出的single-shot、multi-level人臉定位方法,
其基於影像平面之點回歸整合了人臉框預測、2D人臉標示定位及3D頂點回歸。
此模型主要由三個部分組成(
\cref{fig:fig-retinaface-framework}
\fig[0.9][fig:fig-retinaface-framework][!hbt]{fig-retinaface-framework.PNG}[RetinaFace架構][RetinaFace架構]
):
(1)feature pyramid network、(2)context head module及(3)cascade multi-task loss。
首先,feature pyramid network獲得輸入影像,並輸出五個不同比例的特徵圖;
接著,context head module獲得這些特徵圖以計算多任務的損失:
亦即第一個模組會從一般的anchor預測範圍框,
而後第二個模組利用第一個模組迴歸出的anchor以預測更精準的範圍框。
而
\cref{fig:fig-retinaface-ap}
\fig[0.7][fig:fig-retinaface-ap][!hbt]{fig-retinaface-ap.PNG}[RetinaFace(ResNet-152)在WIDER FACE測試集之Precision-Recall曲線][RetinaFace(ResNet-152)在WIDER FACE測試集之Precision-Recall曲線]
展示了RetinaFace和其他29種人臉偵測演算法之平均準確度(Average Precision,AP)數據,
本方法擁有91.7\%的良好結果。


\end{document}