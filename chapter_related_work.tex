\documentclass[class=NCU_thesis, crop=false]{standalone}
\begin{document}

\chapter{相關研究}

\section{嬰兒猝死症}
嬰兒猝死症(The Sudden Infant Death Syndrome, 簡稱SIDS)
~\cite{kinneyTheSuddenInfantDeathSyndrome2009}
之特徵為一位看似健康的嬰兒在睡眠期間突然死亡,
其真正致死之原因尚不明確且非單一。

目前醫界雖未有單一定義此症之直接致死原因,
但可統整出多項促使嬰兒猝死症發生之風險因素,
可分為兩類因素:
其一為外在因素,包含嬰兒因俯臥、側睡或蓋住面部等致使呼吸困難;
其二為內在因素,包含發展因素(如:早產)、
遺傳因素(如:家族性之嬰兒猝死症)、
性別(男性比例為女性的兩倍)或種族等。
除此之外,嬰兒也可能因其他外在環境條件,
如:產前或產後暴露於不良物質中(香菸煙霧、酒精或非法藥物等),
而弱化嬰兒之內在條件。

在嬰兒猝死症研究中,
有許多關於此症之死亡機制理論,
其中心肺控制假說主導了多數研究,
也造就了往後關於嬰兒猝死症之研究多基於嬰兒呼吸或自主神經機制的缺陷。
這樣的論點主要包含了五個步驟:
(1)發生危及生命的事件(如:面部朝下或面部遭遮蔽,將造成反射性或阻塞性呼吸暫停),
而將導致嬰兒窒息、腦部灌注不足或兩者皆發生。
(2)嬰兒無法自行轉頭,以應付窒息的情境,而導致無法從呼吸暫停中恢復。
(3)持續的窒息導致失去意識或反射,即低氧昏迷。
(4)發生心率過緩及缺氧喘氣,此現象在嬰兒因嬰兒猝死症逝世前將明顯發生。
(5)嬰兒之自主復甦能力受損,即因無效的喘氣而最終導致呼吸暫停及死亡。
因此,由嬰兒猝死症之紀錄中,
可看出此症狀並非一種突發疾病,
而是在嬰兒死亡前,
即會出現心率不正常或呼吸暫停之惡性循環現象。

另外,
醫界亦發現俯臥睡姿將會使嬰兒猝死症之風險增加三倍以上,
故在1990年代初期國際間即提倡嬰兒仰臥睡姿,
嬰兒猝死症之發病率也因此降低了50\% 以上,
但仍為嬰兒主要死亡原因之一。

\section{嬰兒監測系統}
在照護嬰兒的過程中,
由於嬰兒尚未發展出語言能力表達自己的不適,
或尚無能力將自己避免於危險之外。
因此,為了協助照顧者關注嬰兒狀態,
現有許多為自動化監測嬰兒之研究,
主要分為以感測器偵測生理訊號及以影像式偵測兩種方式。
% 兩種方式的優缺點比較

\subsection{感測器偵測}
此種方式利用多種不同感測器進行生理訊號之偵測,
包含利用呼吸感測器、濕度感測器、溫度感測器、非接觸式紅外溫度感測器、3D加速度計、慣性感測器等,
分別量測嬰兒之呼吸頻率、出汗狀況、體溫、心率、身體位置或方向、睡眠姿勢、嬰兒周圍的一氧化碳濃度、呼出的二氧化碳濃度的變化等,
且多會透過物聯網技術開發出可穿戴式裝置之系統。

如:Linti等人~\cite{lintiSensoryBabyVestForTheMonitoringOfInfants2006}
所開發的嬰兒感測背心,
其將多個感官元件融入紡織品中以用來量測嬰兒之呼吸、心率、溫度及濕度;
Ferreira等人~\cite{ferreiraASmartWearableSystemForSuddenInfantDeathSyndromeMonitoring2016}
開發了將感測器裝設於胸帶中,
而得以量測嬰兒之體溫、心率、呼吸頻率及身體位置,
並透過ZigBee技術將收集到的數據傳送至伺服器,
用戶則可透過醫療網頁介面進行查看及收到緊急訊息;
Ziganshin等人~\cite{ziganshinUWBBabyMonitor2010}
基於超寬頻技術開發出可監測嬰兒呼吸及心率之系統,
其可檢測嬰兒之睡眠、清醒及警示狀態。

此種利用感測器監測嬰兒的方法,
雖然可直接量測嬰兒之生理訊號以判斷狀態正常與否,
但仍可能因硬體設備之缺陷無法準確量測,
進而有失判斷準確性,
亦或者因嬰兒需額外穿戴裝置而造成不適,
進而影響嬰兒活動或導致更多危險的發生。

% \subsubsection{非侵入式之嬰兒二氧化碳遠端監測系統}
% 該方法為在嬰兒周圍安置一組二氧化碳感測器,
% 收集嬰兒床附近之二氧化碳濃度變化,
% 以監測與嬰兒呼吸問題相關的事件。
% 其優點為較低成本;
% 但缺點為需針對每個感測器進行校準,
% 以免除不同感測器間數值的不一致。

% \subsubsection{非接觸式之嬰兒心肺監測器}
% 其設計了可進行心肺監測的都卜勒雷達系統,
% 以監測嬰兒是否有心跳。
% 不僅改進了使用現有遠端監測心肺功能系統需專業人士設定與操作的侷限性,
% 亦提供較低成本的雷達系統來開發此產品。

% \subsubsection{多感測器之嬰兒無線式監測系統}
% 其利用三軸加速器、溫度感測器及一氧化碳感測器量測嬰兒之睡眠位置、體溫及周圍一氧化碳濃度,
% 再透過Wifi模組將感測器收取到的資訊傳輸至伺服器。

\subsection{影像式偵測}
此種方式利用電腦視覺技術對於嬰兒影像畫面進行偵測,
現有研究中包含了計算嬰兒之呼吸頻率、關注於嬰兒之面部特徵及嬰兒趴睡姿勢偵測。

Fang等人~\cite{fangAVisionBasedInfantRespiratoryFrequencyDetectionSystem2015}
開發了一基於視覺之非接觸式呼吸頻率偵測系統,
先判斷嬰兒是否正在運動(包含頭部、四肢及身體運動,但不包含因呼吸引起的輕微運動),
若未偵測到嬰兒運動,
則系統開始進行呼吸頻率偵測:
首先,透過空間特徵擷取呼吸之候選點;
接著,利用模糊積分技術選擇呼吸點;
最終,得以計算嬰兒的呼吸頻率,進而可判斷嬰兒是否發生呼吸異常之情形。

Liu等人~\cite{liuVideoBasedIoTBabyMonitorForSIDSPrevention2017}
利用夜視攝影機拍攝在嬰兒床內的嬰兒,
並使用MIT所提出之Eulerian Magnification技術,
放大影片中的細微運動以監測拍攝對象之胸部運動,
若經正規化之像素差異值低於設定閥值,
則判斷其呼吸頻率異常,
進而透過手機裝置發出警報。

Gallo等人~\cite{galloMARRSIDS2019}
提出一名為MARRSIDS的模型,
其利用OpenCV之Haar-Like Features偵測嬰兒之面部特徵。
系統透過嬰兒臉部辨識與否及睜眼狀態,
判斷其是否處於危險情境中,
而需發出聲音警示:
若臉部未被偵測,
則認為嬰兒可能位於不良姿勢需發出警示;
而若嬰兒為睜眼狀態,
則代表嬰兒處於清醒狀態,
並非處於風險中。

% 1. 計算呼吸頻率
% 2. 面部(眼睛)特徵(OpenCV - Haar-Like Features)
% 3. 面部被遮擋(multitask Bayesian deep learning)(1. facial occlusion, 2. facial cover, 3. eye openness, and 4. five-point facial landmark detection)
% 4. 嬰兒面朝下趴著睡(CNN)(1. back sleeping, 2. changing from one position to another, and 3. sleeping in a stomach)

% 面部特徵: https://ieeexplore.ieee.org/document/8730261
% 面部被遮擋: https://ieeexplore.ieee.org/document/8803332

% \subsubsection{利用結構光相機之非接觸式嬰兒呼吸頻率監測系統}
% 該方法利用結構光相機可取得距離資訊,
% 使用平面分割偵測嬰兒之胸部區域,
% 再根據胸部的運動計算呼吸頻率。

% \subsubsection{基於視覺計算呼吸頻率}

\section{殘差網路}
ResNet50 ResNet50

\section{面部辨識}
\subsection{DeepFace}
DeepFace DeepFace

\subsection{FaceNet}
FaceNet FaceNet

\end{document}