\documentclass[class=NCU_thesis, crop=false]{standalone}
\begin{document}

\chapter{相關研究}

\section{嬰兒猝死症}
嬰兒猝死症(The Sudden Infant Death Syndrome, 簡稱SIDS)
~\cite{kinneyTheSuddenInfantDeathSyndrome2009}
之特徵為一位看似健康的嬰兒在睡眠期間突然死亡,
其被定義為一種以上疾病之結果。

儘管嬰兒猝死症之成因目前並未有一單一定義,
但當前醫界研究發現俯臥睡姿將會使嬰兒猝死症之風險增加三倍以上,
故在1990年代初期國際間即提倡嬰兒仰臥睡姿。
雖然嬰兒猝死症之發病率因此降低了50\% 以上,
但仍為嬰兒主要死亡原因之一。

導致嬰兒猝死症之成因並非單一,
但可將風險因素可分為外在及內在兩類:
首先,外在因素包含了嬰兒因俯臥、側睡或蓋住面部等致使呼吸困難,
雖然這些原因並非嬰兒致死因素,
但仍提高了嬰兒猝死症的風險;
而內在因素中,
則包含了發展因素(如:早產)、
推論之遺傳因素(如:家族性之嬰兒猝死症)、
男性(比例為2:1)或因種族等。
此外,嬰兒也有可能因其他外在環境條件,
如:產前或產後暴露於不良物質中(香菸煙霧、酒精或非法藥物等),
亦可能弱化嬰兒之內在條件。

在嬰兒猝死症研究中,
有許多關於此症之死亡機制理論,
其中心肺控制假說主導了多數研究,
也造就了往後關於嬰兒猝死症之研究都基於嬰兒呼吸或自主神經機制的缺陷。


\section{嬰兒偵測系統}
現有自動化監測嬰兒之研究中,
多以感測器量測嬰兒狀態,
以達到降低嬰兒猝死症的潛在風險。

\subsection{穿戴式偵測}
心率、呼吸頻率、體溫、身體位置或方向、睡眠姿勢、周圍的一氧化碳濃度、呼出的二氧化碳濃度的變化

% \subsubsection{感測背心}

% \subsubsection{可穿戴物聯網設備}

% \subsubsection{UWB呼吸偵測系統}

% \subsubsection{利用三軸加速度計偵測嬰兒心率}

% \subsubsection{偵測嬰兒心率、呼吸、體溫、位置}

% \subsubsection{非侵入式之嬰兒二氧化碳遠端監測系統}
% 該方法為在嬰兒周圍安置一組二氧化碳感測器,
% 收集嬰兒床附近之二氧化碳濃度變化,
% 以監測與嬰兒呼吸問題相關的事件。
% 其優點為較低成本;
% 但缺點為需針對每個感測器進行校準,
% 以免除不同感測器間數值的不一致。

% \subsubsection{非接觸式之嬰兒心肺監測器}
% 其設計了可進行心肺監測的都卜勒雷達系統,
% 以監測嬰兒是否有心跳。
% 不僅改進了使用現有遠端監測心肺功能系統需專業人士設定與操作的侷限性,
% 亦提供較低成本的雷達系統來開發此產品。

% \subsubsection{多感測器之嬰兒無線式監測系統}
% 其利用三軸加速器、溫度感測器及一氧化碳感測器量測嬰兒之睡眠位置、體溫及周圍一氧化碳濃度,
% 再透過Wifi模組將感測器收取到的資訊傳輸至伺服器。

\subsection{影像式偵測}
1. 計算呼吸頻率

2. 面部(眼睛)特徵(OpenCV - Haar-Like Features)

3. 面部被遮擋(multitask Bayesian deep learning)(1. facial occlusion, 2. facial cover, 3. eye openness, and 4. five-point facial landmark detection)

4. 嬰兒面朝下趴著睡(CNN)(1. back sleeping, 2. changing from one position to another, and 3. sleeping in a stomach)

% 面部特徵: https://ieeexplore.ieee.org/document/8730261
% 面部被遮擋: https://ieeexplore.ieee.org/document/8803332

% \subsubsection{利用結構光相機之非接觸式嬰兒呼吸頻率監測系統}
% 該方法利用結構光相機可取得距離資訊,
% 使用平面分割偵測嬰兒之胸部區域,
% 再根據胸部的運動計算呼吸頻率。

% \subsubsection{基於視覺計算呼吸頻率}

\section{殘差網路}
ResNet50 ResNet50

\section{面部辨識}
\subsection{DeepFace}
DeepFace DeepFace

\subsection{FaceNet}
FaceNet FaceNet

\end{document}