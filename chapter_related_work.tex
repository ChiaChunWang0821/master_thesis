\documentclass[class=NCU_thesis, crop=false]{standalone}
\begin{document}

\chapter{相關研究}

\section{嬰兒猝死症}
嬰兒猝死症 嬰兒猝死症

\section{嬰兒偵測}
\subsection{利用結構光相機之非接觸式嬰兒呼吸頻率監測系統}
該方法利用結構光相機可取得距離資訊,
使用平面分割偵測嬰兒之胸部區域,
再根據胸部的運動計算呼吸頻率。

\subsection{非侵入式之嬰兒二氧化碳遠端監測系統}
該方法為在嬰兒周圍安置一組二氧化碳感測器,
收集嬰兒床附近之二氧化碳濃度變化,
以監測與嬰兒呼吸問題相關的事件。
其優點為較低成本;
但缺點為需針對每個感測器進行校準,
以免除不同感測器間數值的不一致。

\subsection{非接觸式之嬰兒心肺監測器}
其設計了可進行心肺監測的都卜勒雷達系統,
以監測嬰兒是否有心跳。
不僅改進了使用現有遠端監測心肺功能系統需專業人士設定與操作的侷限性,
亦提供較低成本的雷達系統來開發此產品。

\subsection{多感測器之嬰兒無線式監測系統}
其利用三軸加速器、溫度感測器及一氧化碳感測器量測嬰兒之睡眠位置、體溫及周圍一氧化碳濃度,
再透過Wifi模組將感測器收取到的資訊傳輸至伺服器。

\section{殘差網路}
ResNet50 ResNet50

\section{面部辨識}
\subsection{DeepFace}
DeepFace DeepFace

\subsection{FaceNet}
FaceNet FaceNet

\end{document}