\documentclass[class=NCU_thesis, crop=false]{standalone}
\begin{document}

\chapter{Abstract}
When taking care of the infant, the baby caregiver may not be able to pay attention to the status of the baby at any time, 
which may cause the infant to suffer from unpleasant breathing due to overflowing milk, turning over, sleeping on the stomach, etc.

In addition, the existing products use multiple sensors to detect the state of the infant, 
which has a single function and many restrictions on use, resulting in poor convenience.

Therefore, this paper proposes a danger detection system based on deep learning technology, 
focusing on face occlusion and gesture recognition of baby images:
The ResNet50 network is used to train the model to analyze whether the infant's face is obscured by foreign objects 
and recognize four basic postures: lying, lying on the stomach, sitting and standing. 

The recognition accuracy of both parts is 98\%.

Therefore, when the system inputs a infant video, 
the model can recognize that the infant's posture may be in a dangerous state or the face is covered by a foreign object, 
and the caregiver can be immediately alerted.

Since there is currently no public infant data set, 
the infant photos used in this article are all captured and pre-processed from online pictures and videos.

\vspace{2em}
\noindent \textbf{Keywords:} \keywordsEn{} % Set keywords in config.tex
\end{document}