\documentclass[class=NCU_thesis, crop=false]{standalone}
\begin{document}

\chapter{Abstract}
According to the report from the Statistics Department of Taiwan’s Ministry of Health and Welfare,
heart disease has been among the top three causes of death in Taiwan for many years.
Among them, coronary artery disease is one of the most common heart diseases.
Currently, computed tomography (CT) is often used in coronary artery examinations.
By extracting the coronary arteries in the CT image,
the subsequent analysis of vascular stenosis and calcification can be carried out.
However, manual labeling of coronary arteries is a time-consuming task.
Therefore, how to automate the process to reduce the human resources consumed in the process is very important.

In addition, the contrast agent used in computed tomography usually contains iodine.
For some subjects, the contrast agent may cause allergies or damage their kidney function, so those subjects are unsuitable for the contrast agent.
However, the contrast of CT images without contrast enhancement to vessels is very low, and it significantly increases the difficulty of labeling coronary arteries.
Therefore, how to use CT images without contrast enhancement for coronary artery segmentation is also a challenging problem.

This research uses CT images with contrast enhancement to train a model that can automatically segment coronary arteries from CT images and use the results of segmentation to develop applications that perform analysis of calcification location and vascular stenosis.
In addition, this research also used CT images without contrast enhancement for experiments.
This research uses deep learning techniques to implement a data augmentation model to convert the CT image with contrast enhancement into a virtual CT image without contrast enhancement.
The generated virtual CT images are then used to assist in training a model of coronary artery segmentation with CT images without contrast enhancement.

Experimental results show that this research's deep learning model can effectively extract coronary arteries from CT images with contrast enhancement.
For the test CT images with contrast enhancement, the result of Dice Coefficient 0.7868 can be achieved.
The segmentation results can be used to analyze calcification and vascular stenosis, and assist physicians in diagnosis.
For the CT images without contrast enhancement experiment, the test result can be increased from Dice Coefficient 0.5158 to 0.5674 by using the training data generated by the data augmentation model of this study.
The results show the effectiveness of the data augmentation model.

\vspace{2em}
\noindent \textbf{Keywords:} \keywordsEn{} % Set keywords in config.tex
\end{document}