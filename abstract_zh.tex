\documentclass[class=NCU_thesis, crop=false]{standalone}
\begin{document}

\chapter{摘要}
嬰兒照護者在照顧嬰兒時,可能發生無法隨時關注嬰兒狀態的情形,使得嬰兒因溢奶、翻身、趴睡等情形,致使呼吸不順而發生憾事。

又因現有產品多用感測器偵測嬰兒狀態,功能單一且多有使用限制,便利性不佳。

因此,本論文提出基於深度學習技術,專注於嬰兒影像畫面進行臉部遮擋及姿勢辨識之危險偵測系統:
利用 ResNet50 網路訓練模型,以分析嬰兒臉部是否遭異物遮蔽及辨識四種基礎姿勢:正躺、趴躺、坐姿及站立。

兩部分之辨識精確度皆達 98\%。

故當系統輸入嬰兒影片時,可透過模型辨識嬰兒的姿勢可能處於危險狀態或臉部遭異物遮擋,則可即時警示照護者。

由於目前未有公開之嬰兒資料集,故本文中所使用的嬰兒照片皆為網路圖片及影片進行擷取並前處理而成。

% 結果

\vspace{2em}
\noindent \textbf{關鍵字:} \keywordsZh{} % Set keywords in config.tex
\end{document}