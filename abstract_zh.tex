\documentclass[class=NCU_thesis, crop=false]{standalone}
\begin{document}

\chapter{摘要}
照護者在照顧嬰兒時,
可能發生無法隨時關注其狀態的情形,
使嬰兒因溢奶、翻身、趴睡等情形,
致使呼吸不順而發生憾事。
又因現有產品多為感測器式嬰兒偵測系統,
功能單一且易干擾孩童;
而既有的視覺式嬰兒偵測研究中,
又多僅關注於呼吸頻率、面部特徵及單一動作,
尚有許多值得探討之處。

因此,
本論文提出基於深度學習技術,
專注於嬰兒影像畫面之危險監測系統,
包含兩大功能之偵測:
(1)臉部遮擋辨識:
判斷嬰兒臉部是否遭非奶嘴之異物遮蔽,
進而可能發生窒息危險、
及(2)姿勢辨識:
分析嬰兒正躺、爬躺、坐姿及站立四種基礎姿勢,
若為趴躺或站立之姿,
則有可能發生呼吸不順或跌落床面等危險。
綜上功能,
當本系統讀取一段嬰兒影片後,
可藉模型判斷嬰兒是否處於警示狀態,
以提醒照護者。

本研究中,
嬰兒臉部偵測部分,
使用速度較快的SSD演算法,
以及準確率較高的RetinaFace演算法,
使整體系統在執行速度及準確度間達到平衡。
而由於目前未有公開之嬰兒資料集,
故本文收集網路真實嬰兒之不同視角圖片及影片,
自製嬰兒臉部與姿勢資料集各3475張及15416張影像,
再以ResNet50進行臉部遮擋辨識及姿勢辨識兩模型之訓練,
其訓練及測試準確度皆達99\%。
由此證明,
本研究對於嬰兒危險監測系統具有良好的可用性及獨特性。

\vspace{2em}
\noindent \textbf{關鍵字:} \keywordsZh{} % Set keywords in config.tex
\end{document}