\documentclass[class=NCU_thesis, crop=false]{standalone}
\begin{document}

\chapter{摘要}
根據台灣衛生福利部統計處統計,
心臟疾病長年位居台灣前三大死因,
其中冠狀動脈心臟病又是十分常見的心臟疾病之一。
目前在冠狀動脈檢查時,經常使用非侵入性的電腦斷層掃描,
透過於電腦斷層影像中標註冠狀動脈位置,
可以進行後續的血管狹窄度、鈣化位置的分析,
然而以人工方式進行冠狀動脈血管標註非常花費時間,
因此如何自動化流程,以減少過程中所耗費的人力資源,便顯得十分重要。

除此之外,
電腦斷層掃描中使用的顯影劑通常含有碘的成分,
對於部分受檢者可能產生過敏或對其腎臟造成傷害,
因此較不適合使用顯影劑,
然而無顯影劑增強之電腦斷層掃描影像對於血管的對比度較低,
顯著地增加了冠狀動脈血管的標註難度,
因此如何運用無顯影劑增強之電腦斷層影像進行冠狀動脈分割,
也是具有挑戰性的問題。

本研究利用深度學習技術,
使用有顯影劑增強之電腦斷層影像,
訓練能夠自動地從電腦斷層影像中分割冠狀動脈的模型,
並且利用分割之結果開發應用,
進行鈣化位置以及血管狹窄度的分析。
此外本研究也使用無顯影劑增強之電腦斷層影像進行實驗,
使用深度學習技術實現一個資料擴增模型,
將有顯影劑增強之電腦斷層掃描影像轉換為虛擬的無顯影劑增強之電腦斷層影像,
並將產生的影像利用於輔助訓練以無顯影劑增強之電腦斷層影像進行冠狀動脈分割的模型。

實驗結果顯示,
本研究能夠利用有顯影劑增強之電腦斷層影像有效的分割冠狀動脈,
對於測試資料以7-fold交叉驗證能夠達到Dice Coefficient 0.7868的結果,
並能夠以此分割結果進行鈣化位置以及血管狹窄度分析,
給予醫師在診斷上的輔助。
而在無顯影劑增強之電腦斷層影像實驗中,
透過利用本研究的資料擴增方式產生的訓練資料輔助模型訓練,
對於測試資料以5-fold交叉驗證能夠從Dice Coefficient 0.5158增加到0.5674,
顯示了該資料擴增方法的有效性。

\vspace{2em}
\noindent \textbf{關鍵字:} \keywordsZh{} % Set keywords in config.tex
\end{document}