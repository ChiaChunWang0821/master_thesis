\documentclass[class=NCU_thesis, crop=false]{standalone}
\begin{document}

\chapter{摘要}
嬰兒照護者在照顧嬰兒時,
可能發生無法隨時關注嬰兒狀態的情形,
使得嬰兒因溢奶、翻身、趴睡等情形,
致使呼吸不順而發生憾事。
又因現有產品多用感測器偵測嬰兒狀態,
功能單一且多有使用限制,
便利性不佳。

因此,
本論文提出基於深度學習技術,
專注於嬰兒影像畫面之危險監測系統:
首先,
使用SSD演算法及RetinaFace演算法偵測嬰兒臉部;
接著,
利用ResNet50訓練模型,
用以辨識嬰兒臉部是否遭非奶嘴之異物遮蔽,
以及辨識四種基礎姿勢:正躺、趴躺、坐姿及站立。
故當系統輸入嬰兒影片時,
可透過模型辨識嬰兒的姿勢可能處於危險狀態或臉部遭異物遮擋,
則可即時警示照護者。

本文中,
利用SSD演算法偵測之時間優勢,
偵測每張影像平均僅需0.04秒,
且準確度達99\%;
使用RetinaFace演算法偵測嬰兒臉部雖需較長時間,
但其正確率、準確度及召回率皆達99\%;
而危險偵測之臉部遮擋辨識模型及姿勢辨識模型,
其準確度亦皆達99\%。

另外,
由於目前未有公開之嬰兒資料集,
本文中所使用的嬰兒照片皆為網路真實嬰兒圖片及影片,
進行前處理收集而成。

\vspace{2em}
\noindent \textbf{關鍵字:} \keywordsZh{} % Set keywords in config.tex
\end{document}