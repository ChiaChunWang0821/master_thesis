\documentclass[class=NCU_thesis, crop=false]{standalone}
\begin{document}

\chapter{緒論}
\section{研究動機}
根據衛生福利部統計處所發布的嬰兒主要死因統計 [1]中,
101年至105年間每年至少30位嬰兒死於嬰兒猝死症候群(Sudden infant death syndrome,簡稱SIDS),
106年至109年每年亦仍有超過20位嬰兒因此症狀逝世,
是為嬰兒十大死亡原因之一。

三軍總醫院對於嬰兒猝死症的說明為:
一個原本無異狀的嬰兒,突然且無法預期的死亡,
常發生在嬰兒睡眠時,並在事後的屍體解剖檢查中找不到其真正致死原因。
凡未滿一歲的嬰幼兒皆可能發生,其中二至四個月時期尤為常見,
亦可能發生在嬰兒出生一至兩周內。
醫界雖持續探討嬰兒猝死症的發生原因,
但目前對於真正的成因仍不清楚,
綜合醫界當前相關因素的研究中,
包含了嬰兒因溢奶或嘔吐產生呼吸道緊縮反射及憋氣,
或因翻身、趴睡致使呼吸困難,
而窒息死亡等原因。

當照護者在嬰兒照護時,
可能有許多不可避免的情形,
而難免發生視線離開嬰兒的情形,
如:泡奶、做飯、上廁所等,
進而無法百分之百關注嬰兒的各種行為。
而若此時嬰兒發生溢奶、物品遮蓋口鼻、自行翻身或站立等情形,
將造成嬰兒處於危險情境中,
而可能導致憾事發生。

國內外有許多為自動化監測嬰兒狀態之研究,
主要包含兩種偵測方式:
其一為使用感測器量測嬰兒之特定生理訊號,
如:心率、呼吸頻率、體溫、身體位置或方向及嬰兒周圍之氣體濃度等,
透過收集到的數值以判定所監測之嬰兒處於正常狀態與否;
然而,使用此種監測方式具功能單一性,
若欲偵測其他生理訊號,則需增設更多不同種類的感測器,
不僅可能影響嬰兒之活動,亦可能產生更多潛在的危險性,
如:裝置纏繞嬰兒、孩童誤食裝置等。
其二為透過電腦視覺偵測嬰兒影像,
判定嬰兒是否處於危險狀態,
而現有研究中多僅針對嬰兒之面部特徵或單一狀態進行偵測;
然而,我們認為一張嬰兒影像包含了許多資訊得以應用,
如:同時偵測嬰兒面部及姿勢等,
則可透過影像偵測更廣泛的監測嬰兒不同危險情境。

因此,本論文開發出一透過嬰兒影像辨識其基礎姿勢與面部狀態,
以監測嬰兒是否因姿勢不適當或面部遭異物遮擋,
處於危險情境中而需警示照護者。
此方法不僅擁有可監測多種不同危險情境之優點,
亦可減少感測器式偵測將干擾嬰兒之缺點,
且對於未來欲增加其他監測功能有良好的擴充性。

\section{研究目的}
本論文基於深度學習技術,
利用ResNet50網路進行嬰兒動作及臉部遮擋之模型訓練,
且以DeepFace演算法前處理影像擷取出嬰兒臉部畫面,
而得以對嬰兒進行危險監測。

本研究預計達成以下目標:
\begin{itemize}
    \item 針對嬰兒姿勢部分,辨識嬰兒之正躺、趴睡、坐姿及站立之四項基礎姿勢,進而判斷嬰兒是否做出具危險之動作。
    \item 針對嬰兒臉部部分,判斷嬰兒是否因嘔吐物、毛巾等非奶嘴之外物遮蓋其面部,而可能使嬰兒發生窒息危險。
\end{itemize}

綜上目標,本論文將建構出一可對嬰兒姿勢及臉部遮擋進行危險監測之系統。

\section{論文架構}
本論文分為五個章節,其架構如下:

第一章、緒論,敘述本論文之研究動機、研究目的及論文架構。

第二章、相關研究,
敘述嬰兒猝死症之定義,
並探討近年與嬰兒監測相關之研究以及深度學習模型架構與面部辨識網路。

第三章、研究方法,
說明本研究之詳細內容,如:資料集之分類定義及前處理、以及完整系統之流程說明。

第四章、實驗設計與結果,
說明各項實驗設計內容以及評估方法,並對於實驗結果進行探討。

第五章、結論與未來展望,
對於研究結果進行總結,並討論研究的未來展望。

\end{document}

