\documentclass[class=NCU_thesis, crop=false]{standalone}
\begin{document}

\chapter{緒論}
\section{研究動機}
在嬰兒照護時,難免發生照顧者視線離開嬰兒的情形,
如:泡奶、做飯、上廁所等,無法百分之百關注嬰兒的各種行為,
而若此時嬰兒發生溢奶、物品遮蓋口鼻、自行翻身或站立等情形,
對嬰兒具危險性而可能導致憾事發生。

根據衛生福利部統計處所發布的嬰兒主要死因統計 [1]中,
101年至105年間每年至少30位嬰兒死於嬰兒猝死症候群(Sudden infant death syndrome,簡稱SIDS),
106年至109年雖死亡數減少,
但每年仍有超過20位嬰兒因此逝世,為嬰兒十大死亡原因之一。

三軍總醫院對於嬰兒猝死症的說明為:
一個原本無異狀的嬰兒,突然且無法預期的死亡,
常發生在嬰兒睡眠時,並在事後的屍體解剖檢查中找不到其真正致死原因。
凡未滿一歲的嬰幼兒皆可能發生,其中二至四個月時期尤為常見,
亦可能發生在嬰兒出生一兩周內。
然而,目前對於嬰兒猝死症的成因仍不清楚,
綜合醫界目前相關因素的研究中,
包含了嬰兒因溢奶或嘔吐產生呼吸道緊縮反射及憋氣,
或因翻身、趴睡致使呼吸困難,
而窒息死亡等原因。

國內外有許多為自動化監測嬰兒狀態之研究,
大多透過感測器來量測嬰兒之特定危險狀態,
這些監測方式具單一性,
若欲增加其他功能則須裝設更多的感測器,
不僅可能影響嬰兒之活動,亦可能產生更多潛在的危險性,
如:嬰兒誤食裝置、裝置纏繞嬰兒等。

因此,我們認為直接透過攝影機拍攝嬰兒影像畫面,
辨識嬰兒狀態以進行危險偵測,
不但能同時偵測多種不同危險情境,
亦可以減少干擾嬰兒行為,
並免除更多的潛在危險。


\section{研究目的}
本研究利用ResNet50進行嬰兒動作及臉部遮擋之辨識,
且透過DeepFace演算法前處理嬰兒影像以擷取嬰兒臉部影像畫面,
而得以對嬰兒進行危險偵測。

本研究預計達成以下目標:
\begin{itemize}
    \item 針對嬰兒姿勢,判斷嬰兒是否處於趴睡或站立姿勢,而有潛在危險發生。
    \item 針對嬰兒臉部,判斷嬰兒是否因嘔吐物、毛巾等外物遮蓋其口鼻,而可能使嬰兒發生窒息危機。
\end{itemize}

\section{論文架構}
本論文分為五個章節,其架構如下:

第一章、緒論,敘述本論文之研究動機、研究目的及論文架構。

第二章、相關研究,
敘述嬰兒猝死症之定義及現有研究,
並探討近年嬰兒偵測之相關研究、深度學習模型架構及面部辨識網路。

第三章、研究方法,
說明本研究之詳細內容,如:資料集之分類定義及前處理、以及完整系統之流程說明。

第四章、實驗設計與結果,
說明實驗設計內容以及評估方法,並對於實驗結果進行探討。

第五章、結論與未來展望,
對於研究結果進行總結,並討論研究的未來展望。

\end{document}

