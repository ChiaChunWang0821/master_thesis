\documentclass[class=NCU_thesis, crop=false]{standalone}
\begin{document}

\chapter{緒論}
\section{研究動機}
根據衛生福利部統計處所發布的嬰兒主要死因統計 [1]中,
101年至105年間每年至少30位嬰兒死於嬰兒猝死症候群(Sudden infant death syndrome,簡稱SIDS),
106年至109年間雖因此逝世之嬰兒數減少,
但每年仍有超過20位嬰兒因此症死亡,是為嬰兒十大死亡原因之一。

三軍總醫院對於嬰兒猝死症的說明為:
一個原本無異狀的嬰兒,突然且無法預期的死亡,
常發生在嬰兒睡眠時,並在事後的屍體解剖檢查中找不到其真正致死原因。
凡未滿一歲的嬰幼兒皆可能發生,其中二至四個月時期尤為常見,
亦可能發生在嬰兒出生一兩周內。
醫界雖關注於嬰兒猝死症的發生原因,
但目前對於真正的成因仍不清楚,
綜合醫界目前相關因素的研究中,
包含了嬰兒因溢奶或嘔吐產生呼吸道緊縮反射及憋氣,
或因翻身、趴睡致使呼吸困難,
而窒息死亡等原因。

當照護者在嬰兒照護時,
可能有許多不可避免的情形,
而難免發生視線離開嬰兒的情形,
如:泡奶、做飯、上廁所等,
進而無法百分之百關注嬰兒的各種行為。
而若此時嬰兒發生溢奶、物品遮蓋口鼻、自行翻身或站立等情形,
將造成嬰兒處於危險情境中,
而可能導致憾事發生。

國內外有許多為自動化監測嬰兒狀態之研究,
其一為使用感測器量測嬰兒之特定生理訊號,
透過數值之判定來監測嬰兒處於正常狀態,
然而,使用此種監測方式具功能單一性,
若欲監測其他生理訊號,就需增設不同種類的感測器,
不僅可能影響嬰兒之活動,亦可能產生更多潛在的危險性,
如:裝置纏繞嬰兒、孩童誤食裝置等;
其二為透過電腦視覺偵測嬰兒影像,
判定嬰兒是否處於危險狀態,
然而,現有研究中多僅針對嬰兒之面部特徵或單一狀態進行偵測,
但一張嬰兒影像即包含了許多資訊得以應用,
如:嬰兒面部及姿勢等。

因此,本論文透過攝影機直接拍攝嬰兒影像,
以辨識嬰兒面部及姿勢狀態而進行危險監測,
此方法不但擁有可彈性應用在不同危險情境之優點,
亦可減少感測器式偵測將干擾嬰兒之缺點。

\section{研究目的}
本研究利用ResNet50進行嬰兒動作及臉部遮擋之辨識,
並透過DeepFace演算法前處理影像以擷取嬰兒臉部畫面,
而得以對嬰兒進行危險偵測。

本研究預計達成以下目標:
\begin{itemize}
    \item 針對嬰兒姿勢,辨識嬰兒之正躺、趴睡、坐姿及站立之四項基礎姿勢,進而判斷嬰兒是否處於趴睡或站立姿勢,而有潛在危險發生。
    \item 針對嬰兒臉部,判斷嬰兒是否因嘔吐物、毛巾等非奶嘴之外物遮蓋其面部,而可能使嬰兒發生窒息危險。
\end{itemize}

\newpage
\section{論文架構}
本論文分為五個章節,其架構如下:

第一章、緒論,敘述本論文之研究動機、研究目的及論文架構。

第二章、相關研究,
敘述嬰兒猝死症之定義,
並探討近年嬰兒偵測之相關研究以及深度學習模型架構與面部辨識網路。

第三章、研究方法,
說明本研究之詳細內容,如:資料集之分類定義及前處理、以及完整系統之流程說明。

第四章、實驗設計與結果,
說明各項實驗設計內容以及評估方法,並對於實驗結果進行探討。

第五章、結論與未來展望,
對於研究結果進行總結,並討論研究的未來展望。

\end{document}

