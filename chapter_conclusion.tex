\documentclass[class=NCU_thesis, crop=false]{standalone}
\begin{document}

\chapter{總結}
\section{結論}
本研究分別利用了有顯影劑增強以及無顯影劑增強之電腦斷層影像進行冠狀動脈分割實驗,
結果顯示,在有顯影劑增強之電腦斷層影像中,
深度學習模型已能十分有效的將冠狀動脈分割任務完成,
並且能夠提供良好的分割結果以進行相關應用。

對於無顯影劑增強影像之冠狀動脈分割任務,
深度學習模型對於冠狀動脈分割也能達到一個初步的成果,
取得大致的冠狀動脈主要分布,
對於醫師在缺少有顯影劑增強的資料,
如受檢者因為身體因素無法接受顯影劑注射時,
能夠做為額外的輔助診斷資訊。

本研究將既有的已標記之有顯影劑增強影像,
以CycleGAN進行影像的風格轉換,產生虛擬的無顯影劑增強影像,
做為無顯影劑增強影像冠狀動脈分割任務的額外訓練資料,
並且有效地輔助無顯影劑增強資料進行訓練,使得模型結果有所提升。

最後本研究也以有顯影劑增強之影像所分割出的冠狀動脈結果,
以3D Slicer插件的形式,實作鈣化位置偵測以及狹窄度分析的應用,
提升了有顯影劑增強影像及其冠狀動脈分割結果在應用上的價值。

\section{未來展望}
本研究尚有一些能夠改進的部分,期望能在未來繼續研究進行改善與加強。

在冠狀動脈分割模型的部分,目前對於無顯影劑資料進行分割的結果尚還不如有顯影劑的資料,
且目前的訓練資料數量還是不足以提供很好的多樣性,
導致對於不同的樣本模型效果差異較大,
此外本研究在模型及資料前處理的部分也較為簡單,
或許未來能透過使用更複雜的模型、更複雜的前處理方法,
例如取得更多的原始影像特徵,來提升冠狀動脈分割模型的效果。

在CycleGAN的模型訓練過程中,
如何以量化的方式評估CycleGAN進行風格轉換的效果,
目前尚是一個未解決的問題,或許未來能設計一個評估的演算法,
如利用邊緣偵測方式評估心臟結構是否有被改變,
以及利用影像HU值分布範圍來評估目前的影像為有、無顯影劑增強,
使得CycleGAN結果能夠更加精確地被評估。

最後在相關應用的部分,
目前提供的方法是以輔助醫師診斷為主,尚未能夠提供較全面的自動化流程,
在鈣化位置偵測方面,目前需要手動輸入HU值範圍以進行鈣化位置的偵測,
未來或許能透過無顯影劑資料預先計算受檢者的鈣化分數,
並將其與有顯影劑資料偵測之鈣化位置計算的鈣化分數進行比較,
或做為模型訓練的目標,訓練一個能直接找尋鈣化位置的模型,
進而直接提供一個較準確的初始結果,減少手動調整的需求。
而在狹窄度偵測方面,由於目前沒有取得血管狹窄相關的標註資料,
因此尚未能直接對於血管狹窄的狀況進行偵測,
而是僅能提供拉直後的血管影像、管徑趨勢等輔助判斷資料,
若未來能取得相關標註資料,或許便能進一步以深度學習方式來解決此問題。



\end{document}