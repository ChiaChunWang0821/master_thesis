\documentclass[class=NCU_thesis, crop=false]{standalone}
\begin{document}

\chapter{結論與未來展望}
\section{結論}
本論文基於深度學習技術,
透過嬰兒影像畫面進行危險偵測,
目前可進行兩大部分之偵測:
(1)嬰兒臉部遮擋辨識、及(2)嬰兒姿勢辨識,
其訓練及測試準確率皆達98\%以上。

本系統優於過往感測器式偵測之功能單一性及不便性,
也不同於既有之影像式偵測僅關注嬰兒呼吸或單一動作之研究,
而提供了關注於嬰兒臉部及動作之危險監測系統,
將有助於協助照護者,
並降低嬰兒猝死症發生風險。

另外,
由於目前未有公開之嬰兒資料集,
故本論文使用之所有嬰兒影像,
皆收集自網路上實際嬰兒照片或影片擷取,
再經前處理及分類標示而成。

\newpage

\section{未來展望}
本論文中,
目前僅辨識四項嬰兒姿勢,
若在偵測姿勢時加入時間資訊,
預期得以判斷更多嬰兒行為,
如:翻身及爬行等動作,
即可監測更多危險情境;
而除辨識嬰兒臉部遭異物遮蔽外,
若加入偵測面部表情等其他資訊,
亦可更詳盡監測嬰兒狀態,
以提醒照護者;
此外,
亦可提供多嬰兒情境之危險偵測,
則使用場景將可更廣泛。

而系統實作方面,
未來可提供設定觀測之年齡區間,
即可針對不同之特定年齡嬰幼兒警示其具危險性之動作,
以達到更符合實際使用情境的危險偵測;
亦可結合通訊社群軟體等,
如:Line或Telegram等,
進行即時之推播訊息以通知照顧者。

\end{document}