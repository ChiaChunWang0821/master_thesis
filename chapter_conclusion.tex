\documentclass[class=NCU_thesis, crop=false]{standalone}
\begin{document}

\chapter{結論與未來展望}
本章節中,
對本論文進行總結,
並闡述本研究未來可發展與應用之處,
將透過以下兩章子節進行分述:
結論、未來展望。

\section{結論}
本論文基於深度學習技術,
透過嬰兒影像畫面進行危險偵測,
目前可進行兩大功能之偵測:
(1)嬰兒臉部遮擋辨識、及(2)嬰兒姿勢辨識。

本系統優於過往感測器式偵測之功能單一性及不便性,
也不同於既有之影像式偵測僅關注嬰兒呼吸或單一動作之研究,
而提供了關注於嬰兒臉部及動作之危險監測系統,
將有助於協助照護者,
並降低嬰兒猝死症發生風險。

\newpage

\section{未來展望}
本系統目前僅針對單一嬰兒之情境進行辨識,
未來可提供多嬰兒情境之危險偵測,
則使用場景將可更廣泛。

姿勢辨識部分,
目前僅辨識四種動作,
若在偵測姿勢時加入時間資訊,
預期得以判斷更多嬰兒行為,
如:翻身及爬行等動作。

而臉部遮擋辨識部分,
由於資料集影像多為嬰兒正臉,
若增加側臉及更多樣情境之照片,
將有助於改善臉部遮擋辨識誤判問題;
亦也可增加面部表情等其他資訊,
則可更詳盡的監測嬰兒狀態。

另外,
系統設計方面,
未來可提供設定觀測之年齡區間,
針對不同之特定年齡嬰幼兒警示其具危險性之動作,
以達到更符合實際使用情境的危險偵測。

\end{document}