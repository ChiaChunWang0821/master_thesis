\documentclass[class=NCU_thesis, crop=false]{standalone}
\begin{document}

\chapter{背景知識以及文獻回顧}

\section{背景知識}
\subsection{電腦斷層掃描}
電腦斷層掃描(Computed Tomography, CT)~\cite{nadrljanskiComputedTomographyRadiology,mckavanaghEssentialsCardiacComputerized2015}是目前廣泛運用的一種非侵入性醫療檢查,
與X射線相同,電腦斷層掃描以測量組織的密度進行成像,
因此影像中越亮的部分代表密度越高,反之則代表密度越低。
透過將電腦斷層掃描的二維影像進行堆疊可以獲得身體的三維結構,
使得醫師能夠利用非侵入性的方式取得受檢者身體內部的結構。

運用電腦斷層掃描於心臟冠狀動脈檢查時,
通常會搭配顯影劑的使用,將含碘的顯影劑注射至受檢者的血液,
以增加電腦斷層影像中血液流經處與其他組織間的對比度。
然而,目前已知電腦斷層顯影劑具有腎毒性~\cite{andreucciUpdateRenalToxicity2017},
腎功能不全者、糖尿病使用Metformin成分藥物者~\cite{rasuliMetforminContrastMedia1998}
皆不適合使用。
此外,電腦斷層顯影劑有5-8\%的人可能會過敏~\cite{saljoughianIntravenousRadiocontrastMedia2012},
症狀從輕症的潮紅、嘔吐,到致命的休克都有。
此類受檢者不適合拍攝有顯影劑增強之電腦斷層影像,
因此如何達成運用未注射顯影劑的電腦斷層影像進行血管分割是本研究的目標之一。

\subsection{Hounsfield Units}
Hounsfield Units(HU)~\cite{greenwayHounsfieldUnitRadiology,levCTAngiographyCT2002,murphyWindowingCTRadiology}為電腦斷層掃描影像標準化後的數值單位,
是由測量後的衰減係數經過線性轉換後獲得,
其中在標準溫度以及標準壓力下,
蒸餾水被定義為0 HU,空氣被定義為-1000 HU,
HU值計算的公式為\cref{eqn:hu},
其中 $\mu _{tissue}$為測量組織的衰減係數、$\mu _{water}$為蒸餾水的衰減係數。
\begin{equation}
\label{eqn:hu}
HU = 1000 \times (\mu _{tissue}-\mu _{water})/\mu _{water}
\end{equation}

\cref{table:table-hu-of-common-tissues-from-head}為頭部電腦斷層掃描中,常見組織對應的HU值:
\begin{table}[h]
    \centering
    \caption{頭部電腦斷層掃描常見組織之HU值對照表~\cite{levCTAngiographyCT2002}}
    \label{table:table-hu-of-common-tissues-from-head}
    \begin{tabular}{cc}
    \hline
    Hounsfield units & Tissue \\
    \hline
    >1000 & Bone, calcium, metal \\
    100 to 600 & Iodinated CT contrast \\
    30 to 500 & Punctate calcifications \\
    60 to 100 & Intracranial hemorrhage \\
    35 & Gray matter \\
    25 & White matter \\
    20 to 40 & Muscle, soft tissue \\
    0 & Water \\
    -30 to -70 & Fat \\
    <-1000 & Air \\
    \hline
    \end{tabular}
\end{table}

此外,由於HU值的範圍通常十分龐大,導致原始影像對比度較差,
因此在觀察電腦斷層影像時,通常會針對不同觀察目標設定對應的上下界,
將下界之HU值視為最暗、上界之HU值視為最亮,
使得能夠更清楚的觀察目標,
\cref{table:table-hu-common-setting-from-head}為頭部電腦斷層掃描中,常見預設HU上下界設定值,
以軟組織(Soft tissue)為例,
中心點為0 HU、寬度為350 HU,即是下界為-175 HU,上界為175 HU,
不過\cref{table:table-hu-common-setting-from-head}僅為參考之預設值,
實際使用時會依照需求進行細部調整。

\begin{table}[h]
    \centering
    \caption{頭部電腦斷層掃描常見之HU值Window預設值~\cite{levCTAngiographyCT2002}}
    \label{table:table-hu-common-setting-from-head}
    \begin{tabular}{lcc}
    \hline
    Setting name & Center level(HU) & Window width(HU) \\
    \hline
    Routine head & 30 & 80 \\
    Acute stroke & 30 & 300 \\
    Skull & 250 & 4000 \\
    Subdural hematoma & 65 & 130 \\
    CTA & 150 & 450 \\
    Soft tissue & 0 & 350 \\
    Lung & -500 & 1500 \\
    \hline
    \end{tabular}
\end{table}


\subsection{心臟冠狀動脈結構}
心臟冠狀動脈~\cite{ogobuiroAnatomyThoraxHeart2021, rehmanPhysiologyCoronaryCirculation2021}結構如\cref{fig:fig-coronary-artery},
冠狀動脈與主動脈(Aorta)相連接並可以分為左右兩邊,
左邊以左主冠狀動脈(Left Main Coronary Artery, LM)開始,
並可以再分為左前降支(Left Anterior Descending, LAD)以及左迴旋支(Left Circumflex, LCx);
右邊則是以右冠狀動脈(Right Coronary Artery, RCA)為主。

\fig[0.5][fig:fig-coronary-artery][!hbt]{fig-coronary-artery.jpg}[心臟冠狀動脈結構~\cite{ogobuiroAnatomyThoraxHeart2021}][心臟冠狀動脈結構]

冠狀動脈用來供應心肌含氧血,是維持心臟功能的重要血管,
當動脈粥樣硬化(Atherosclerosis)~\cite{AtherosclerosisAmericanHeart, AtherosclerosisNHLBINIH}發生在冠狀動脈時被稱為冠狀動脈心臟病,
可能會導致心絞痛或心臟病的發生,
當冠狀動脈阻塞時會導致心肌梗塞~\cite{ojhaMyocardialInfarction2021},
使得心肌無法獲得氧氣而產生缺氧,影響心臟運作功能。

\subsection{3D U-Net}
3D U-Net~\cite{cicek3DUNetLearning2016}是由Çiçek等人於2016年提出的3D版本U-Net,
架構與Ronneberger等人於2015提出之U-Net大致相同,
差異處為將輸入改為3D影像,並增加了卷積層的Kernel數量以及Batch Normalization,
使得模型效果以及收斂速度能夠有所提升。\cref{fig:3dunet-structure}為3D U-Net的架構,
\fig[][fig:3dunet-structure][!hbt]{fig-3dunet-structure.jpg}[3D U-Net模型架構~\cite{cicek3DUNetLearning2016}][3D U-Net]

\subsection{CycleGAN}
CycleGAN~\cite{zhuUnpairedImagetoImageTranslation2017}是由Zhu等人於2017年提出的模型,
為一個能夠透過非對稱資料集進行訓練,實現圖像生成、風格轉換的模型。
模型架構如\cref{fig:cyclegan-structure},
其特色為有兩組Generator以及Discriminator,
用以將Domain X與Domain Y的影像互相轉換,
並使用cycle-consistency loss避免模型在風格轉換時忽略原始影像的資訊。
\fig[][fig:cyclegan-structure][!hbt]{fig-cyclegan-structure.jpg}[CycleGAN模型架構~\cite{zhuUnpairedImagetoImageTranslation2017}][CycleGAN]

本研究使用CycleGAN進行資料擴增,
訓練一個能夠將有顯影劑增強之電腦斷層影像轉換為無顯影劑增強之影像的模型,
藉以增強無顯影劑影像之血管分割任務的模型效果。

\section{文獻回顧}
\subsection{對於有顯影劑增強之電腦斷層影像進行冠狀動脈分割之研究}
Moeskops等人~\cite{moeskopsDeepLearningMultitask2016}
的研究使用2D CNN對於多種醫學影像任務進行分割,
其中包含了對於有顯影劑電腦斷層影像進行冠狀動脈分割,
其以目標體素(Voxel)中心,取3個不同解剖平面51*51影像大小的Patch做為CNN輸入,
預測該目標體素是否為冠狀動脈,
並在測試資料中達到Dice Coefficient約0.65的結果。

Huang等人~\cite{huangCoronaryArterySegmentation2018}
利用3D U-Net進行有顯影劑增強之心臟電腦斷層影像進行冠狀動脈分割,
實驗了兩種不同有、無中心線資料集進行冠狀動脈分割的差異,
分別使用32*32*32的Patch以及64*64*16的Patch做為3D U-Net輸入,
並在有中心線資料集獲得較好的結果,
對於測試資料達到Dice Coefficient約0.7755的結果。

Chen等人~\cite{chenCoronaryArterySegmentation2019}
利用多通道的3D U-Net進行冠狀動脈分割,
輸入通道包含原始的影像以及利用血管增強濾鏡所取得的影像,
且影像大小32*32*32的Patch,
該研究將輸出結果利用兩次取Largest Connected Component(LCC),
藉此過濾出右冠狀動脈以及左冠狀動脈,減少一些錯誤分割的結果,
對於測試資料集達到Dice Coefficient 0.8060的結果。

由文獻可以得知,
目前已有許多對有顯影劑增強之電腦斷層影像進行冠狀動脈分割的研究,
且在使用3D CNN模型的情況下能夠獲得較佳的結果,
部分研究透過取得原始影像之額外特徵如中心線、血管增強濾鏡,
或是使用後處理方法來增強冠狀動脈的分割結果。

\subsection{對於無顯影劑增強之電腦斷層影像進行分割之研究}
由於目前較少看到對於無顯影劑電腦斷層影像進行冠狀動脈分割的研究,
因此本研究參考了一些非冠狀動脈分割的無顯影劑影像相關研究。

Shahzad等人~\cite{shahzadAutomaticSegmentationQuantification2017}
的研究使用了multi-atlas-based的分割方法,
利用有顯影劑電腦斷層影像標記資料,
並利用套合(Registration)方式對於無顯影劑電腦斷層影像進行心臟結構如全心臟、主動脈、左右心房及心室的分割,
對於測試資料達到Dice Coefficient約0.9的結果。

Patel等人~\cite{patelIntracerebralHaemorrhageSegmentation2019}
的研究使用了使用兩組無顯影劑增強之腦部電腦斷層影像資料,
並使用多通道的3D CNN對於腦部電腦斷層影像進行腦內出血的分割,
對於測試資料中達到Dice Coefficient約0.91的結果。

Tuladhar等人~\cite{tuladharAutomaticSegmentationStroke2020}
的研究使用了無顯影劑增強之腦部電腦斷層影像,
並利用3D CNN模型進行腦中風病灶的分割,
在前處理時去除了腦部骨頭的結構,
並以Connected Component分析去除較小的雜訊進行後處理,
對於測試資料達到Dice Coefficient約0.45的結果。

由文獻可以得知,
利用無顯影劑資料在部分分割研究中也能達到不錯的結果,
然而目前所看到的文獻是對於較大型結構如心臟、出血病灶進行分割,
與本研究對比度差且標記結構較小的冠狀動脈有些差距,
相較之下本研究的分割目標較為困難,因此也較難達到上述研究的結果。

\subsection{將GAN應用於醫學影像之研究}
近年也有一些研究使用GAN模型對於醫學影像進行風格轉換,
並應用於不同場景:

Jiang等人~\cite{jiangTumorAwareAdversarialDomain2018}
運用CycleGAN將電腦斷層掃描影像轉換為核磁共振影像,
使用了考慮腫瘤的損失函數,使得Cycle-GAN轉換時能夠更好的保留腫瘤的資訊,
並利用真實資料以及轉換出的虛擬核磁共振影像做為訓練資料集,
訓練U-Net模型進行腫瘤分割,
使得對於測試資料由僅使用原始資料Dice Coefficient約0.55的結果提升至Dice Coefficient約0.80的結果。

Welander等人~\cite{welanderGenerativeAdversarialNetworks2018}
的研究比較了兩種GAN模型CycleGAN以及UNIT對於T1、T2加權的MRI影像進行風格轉換的差異,
並對於不同模型、資料的誤差進行分析及討論。

Song等人~\cite{songNoncontrastCTLiver2020}
的研究以無顯影劑電腦斷層影像進行肝臟的分割,
其利用CycleGAN將有顯影劑的電腦斷層影像轉換為無顯影劑的電腦斷層影像做為額外的訓練資料,
並使用3D U-Net訓練肝臟分割模型,
使得對於測試資料由僅使用原始資料Dice Coefficient約0.8826的結果提升至Dice Coefficient約0.9420的結果。

以上文獻顯示了利用GAN進行醫學影像風格轉換的可能,
由於本研究中已標記冠狀動脈之無顯影劑電腦斷層影像資料較少,
因此本研究透過GAN產生更多的無顯影劑影像進行訓練,做為提升無顯影劑冠狀動脈模型效果的方法。

\end{document}