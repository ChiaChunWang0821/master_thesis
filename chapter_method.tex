\documentclass[class=NCU_thesis, crop=false]{standalone}
\begin{document}

\chapter{研究方法}

\section{系統流程介紹}
本論文所開發之嬰兒危險偵測系統,
其針對嬰兒影像畫面進行識別,
以判斷嬰兒是否處於危險狀態,
而須提醒照護者。

系統之完整流程為:
首先,輸入一段待觀測之嬰兒影片,
將影片切成數幀影像,
若此影像存在,
則開始進行危險偵測;
每幀待偵測之影像畫面,
將針對嬰兒之臉部是否遮擋及姿勢分別進行辨識,
若透過模型分析為警示狀態,
則再經後續步驟判斷是否提醒照護者;
而若分析為安全狀態,
則可接續下一幀之影像進行偵測。

詳細之系統流程圖,請見\cref{fig:fig-system-flow-chart}。
\fig[0.6][fig:fig-system-flow-chart][!hbt]{fig-system-flow-chart.png}[系統流程圖][系統流程圖]

而本系統中,兩核心模型之辨識步驟如下:
(1) 嬰兒臉部遮擋辨識:
先將嬰兒畫面擷取出僅含臉部範圍之影像,
再透過嬰兒臉部遮擋模型判斷嬰兒臉部是否遭異物遮擋,
並辨識遮蔽物是否為奶嘴,
則可得出嬰兒臉部是否遭異物遮擋,
而須警示照護者;
(2) 嬰兒姿勢辨識:
將嬰兒影像透過嬰兒姿勢模型進行辨識,
判斷嬰兒為安全姿勢:正躺或坐姿,
或為具危險性的姿勢:趴躺或站立,
若辨識為具危險性的姿勢則需警示照護者。

此二部分辨識之詳細方法,
將於第四章及第五章進行介紹。

在實際情境中,
由於嬰兒做出危險行為時,
持續一段時間才會導致危險的發生,
因此我們不須判斷一張畫面為警示狀態,
就立即通知照護者。
故本系統使用一變數累積結果為警示之幀數,
當此變數超過一定值時,
系統才會真正發出警示,
提醒照護者須注意嬰兒之狀態。
此步驟不但更符合實際使用情境,同
時亦可減少因模型辨識錯誤而誤判的情況。

\section{臉部遮擋辨識}
如前言所述,
目前醫界對於嬰兒猝死症之相關因素研究中,
判斷嬰兒臉部是否遭遮蔽,
將有助於降低嬰兒猝死症風險。

此外,亦有研究發現嬰兒使用奶嘴,
對於預防嬰兒猝死症有幫助,
故我們會將嬰兒使用奶嘴之情形排除。

本文對於嬰兒臉部遮擋分成兩步驟辨識:
首先,判斷嬰兒臉部是否有異物,
若臉部無異物則判斷為安全,
反之則為警示;
若判斷為後者,
則會接續判斷此遮蔽物是否為奶嘴,
若為非奶嘴之異物,
將須警示照護者。

\subsection{資料集前處理}
由於此部分辨識僅關注嬰兒臉部影像,
故我們先將收集到的嬰兒畫面透過Deepface演算法~\cite{taigmanDeepFace2014}進行前處理,
以獲得只涵蓋嬰兒臉部影像之資料集。

\subsection{資料集詳細介紹}
我們會將嬰兒影像分為無遮蔽、有遮蔽但遮蔽物為奶嘴及有遮蔽但遮蔽物非奶嘴,
前兩類判斷為安全,
最後一類則為警示狀態。

對於臉部遮擋資料集之三項分類介紹如下:

(1)安全:嬰兒臉部五官未被遮擋,如\cref{fig:fig-face-uncovered}。
\fig[0.6][fig:fig-face-uncovered][!hbt]{fig-face-uncovered.png}[嬰兒臉部無遮蔽][嬰兒臉部無遮蔽]

(2)臉部遮蔽物為奶嘴:嬰兒正在使用奶嘴為安全狀態,如\cref{fig:fig-covered-by-pacifier}。
\fig[0.6][fig:fig-covered-by-pacifier][!hbt]{fig-covered-by-pacifier.png}[嬰兒臉部遮蔽物為奶嘴][嬰兒臉部遮蔽物為奶嘴]

(3)臉部遮蔽物非奶嘴:嬰兒臉部被嘔吐物、溢奶或其他外物遮蔽,可能造成窒息危險,如\cref{fig:fig-covered-by-foreign-matter}。
\fig[0.6][fig:fig-covered-by-foreign-matter][!hbt]{fig-covered-by-foreign-matter.png}[嬰兒臉部遭異物遮擋][嬰兒臉部遭異物遮擋]

嬰兒臉部資料集包含嬰兒之正臉及側臉,共7461張照片,
我們將其分為訓練、測試及預測集,各部分占比為70\%、20\%及10\%。

\subsection{模型訓練}
使用ResNet50網路訓練臉部遮擋模型與奶嘴辨識模型,
訓練回合數皆為20。

\section{姿勢辨識}
承前言所述,
除了臉部遮蔽可能造成嬰兒猝死症外,
嬰兒做出不適當的姿勢也常為危險發生之原因,
如:嬰兒側躺或趴睡時,
因頸部肌肉較弱等原因,
無力自行將臉移開,
造成呼吸困難而窒息死亡;
或者當嬰兒自行站立,
而有可能爬落嬰兒床等,
亦可能使嬰兒處於危險情境中。

\subsection{資料集分類定義}
在實際情況下,
嬰兒姿勢多變且不固定,
而有些動作則需要時間資訊才得以判斷,
如:從正躺移至趴躺或坐姿時,
會做出側躺、翻身的動作;
從趴躺移至坐姿或站立時,
嬰兒的著地點有可能包含手掌、手肘、膝蓋或腳掌等。

一開始我們將姿勢辨識分為五類,
分別為正躺、趴躺、爬行、坐姿及站立,
而其中趴躺及爬行兩類常有互相誤判的結果。
推測原因為動作分類過細,
而導致辨識錯誤率較高。

故最終本論文將嬰兒基礎姿勢分成四類,
包含了正躺(腹面朝上)、趴躺(腹面朝下)、坐姿及站立,
以供辨識嬰兒大部分之姿勢。

對於此四項姿勢之詳細分類定義為:

(1)正躺:嬰兒腹部面朝上,背部貼於水平面,而頭部及手腳位置不限,如\cref{fig:fig-lie-down-on-back}。
\fig[0.8][fig:fig-lie-down-on-back][!hbt]{fig-lie-down-on-back.png}[嬰兒正躺姿勢][嬰兒正躺姿勢]

(2)趴躺:嬰兒腹部面朝下,包含趴著或爬行等多動作,而頭部及手腳位置不限,如\cref{fig:fig-lie-down-on-stomach}。
\fig[0.8][fig:fig-lie-down-on-stomach][!hbt]{fig-lie-down-on-stomach.png}[嬰兒趴躺姿勢][嬰兒趴躺姿勢]

(3)坐姿:嬰兒屁股部位貼於水平面,而背部未貼於同一平面,頭部及手腳位置不限,如\cref{fig:fig-sit}。
\fig[0.8][fig:fig-sit][!hbt]{fig-sit.png}[嬰兒坐姿姿勢][嬰兒坐姿姿勢]

(4)站立:嬰兒腳掌貼於水平面,且腹部和背部皆未平行於此水平面,而頭部及手部位置不限,如\cref{fig:fig-stand}。
\fig[0.8][fig:fig-stand][!hbt]{fig-stand.png}[嬰兒站立姿勢][嬰兒站立姿勢]

而為了能有較廣泛的使用情境,
所收集的嬰兒影像不限定拍攝視角,
包含俯視、平視等,共15416張照片。
並將資料集分為訓練、測試及預測集,
各部分占比為70\%、25\%及5\%。

\subsection{模型訓練}
使用ResNet50網路訓練嬰兒姿勢辨識模型,
訓練回合數為20。

\end{document}