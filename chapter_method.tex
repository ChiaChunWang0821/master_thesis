\documentclass[class=NCU_thesis, crop=false]{standalone}
\begin{document}

\chapter{研究方法}

\section{系統流程介紹}
本論文所開發之嬰兒危險偵測系統,
其針對嬰兒影像畫面進行識別,
以判斷嬰兒是否處於危險狀態,
而須提醒照護者。

系統之完整流程為:
首先,輸入一段待觀測之嬰兒影片,
將影片切成數幀影像,
若此影像存在,
則開始進行危險偵測;
每幀待偵測之影像畫面,
將針對嬰兒之臉部是否遮擋及姿勢分別進行辨識,
若透過模型分析為警示狀態,
則再經後續步驟判斷是否提醒照護者;
而若分析為安全狀態,
則可接續下一幀之影像進行偵測。

詳細之系統流程圖,請見圖\cref{fig:fig-system-flow-chart}。
\fig[0.6][fig:fig-system-flow-chart][!hbt]{fig-system-flow-chart.png}[系統流程圖][系統流程圖]

而本系統中,兩核心模型之辨識步驟如下:
(1) 嬰兒臉部遮擋辨識:
先將嬰兒畫面擷取出僅含臉部範圍之影像,
再透過嬰兒臉部遮擋模型判斷嬰兒臉部是否遭異物遮擋,
並辨識遮蔽物是否為奶嘴,
則可得出嬰兒臉部是否遭異物遮擋,
而須警示照護者;
(2) 嬰兒姿勢辨識:
將嬰兒影像透過嬰兒姿勢模型進行辨識,
判斷嬰兒為安全姿勢:正躺或坐姿,
或為具危險性的姿勢:趴躺或站立,
若辨識為具危險性的姿勢則需警示照護者。

此二部分辨識之詳細方法,
將於第四章及第五章進行介紹。

在實際情境中,
由於嬰兒做出危險行為時,
持續一段時間才會導致危險的發生,
因此我們不須判斷一張畫面為警示狀態,
就立即通知照護者。
故本系統使用一變數累積結果為警示之幀數,
當此變數超過一定值時,
系統才會真正發出警示,
提醒照護者須注意嬰兒之狀態。
此步驟不但更符合實際使用情境,同
時亦可減少因模型辨識錯誤而誤判的情況。

\section{臉部遮擋辨識}
如前言所述,
目前醫界對於嬰兒猝死症之相關因素研究中,
判斷嬰兒臉部是否遭遮蔽,
將有助於降低嬰兒猝死症風險。

此外,亦有研究發現嬰兒使用奶嘴,
對於預防嬰兒猝死症有幫助,
故我們會將嬰兒使用奶嘴之情形排除。

本文對於嬰兒臉部遮擋分成兩步驟辨識:
首先,判斷嬰兒臉部是否有異物,
若臉部無異物則判斷為安全,
反之則為警示;
若判斷為後者,
則會接續判斷此遮蔽物是否為奶嘴,
若為非奶嘴之異物,
將須警示照護者。

\subsection{資料集前處理}
由於此部分辨識僅關注嬰兒臉部影像,
故我們先將收集到的嬰兒畫面透過Deepface演算法~\cite{taigmanDeepFaceUsing2014}進行前處理,
以獲得只涵蓋嬰兒臉部影像之資料集。

\subsection{資料集詳細介紹}
我們會將嬰兒影像分為無遮蔽、有遮蔽但遮蔽物為奶嘴及有遮蔽但遮蔽物非奶嘴,
前兩類判斷為安全,
最後一類則為警示狀態。

對於臉部遮擋資料集之三項分類介紹如下:
(1)安全:嬰兒臉部五官未被遮擋,如圖\cref{fig:fig-face-uncovered}。
\fig[0.6][fig:fig-face-uncovered][!hbt]{fig-face-uncovered.png}[嬰兒臉部無遮蔽][嬰兒臉部無遮蔽]

(2)臉部遮蔽物為奶嘴:嬰兒正在使用奶嘴為安全狀態,如圖\cref{fig:fig-covered-by-pacifier}。
\fig[0.6][fig:fig-covered-by-pacifier][!hbt]{fig-covered-by-pacifier.png}[嬰兒臉部遮蔽物為奶嘴][嬰兒臉部遮蔽物為奶嘴]

(3)臉部遮蔽物非奶嘴:嬰兒臉部被嘔吐物、溢奶或其他外物遮蔽,可能造成窒息危險,如圖\cref{fig:fig-covered-by-foreign-matter}。
\fig[0.6][fig:fig-covered-by-foreign-matter][!hbt]{fig-covered-by-foreign-matter.png}[嬰兒臉部遭異物遮擋][嬰兒臉部遭異物遮擋]

嬰兒臉部資料集包含嬰兒之正臉及側臉,共7461張照片,
我們將其分為訓練、測試及預測集,各部分占比為70\%、20\%及10\%。

\subsection{模型訓練}
模型訓練 模型訓練

\section{姿勢辨識}
\subsection{資料集分類定義}
資料集分類定義 資料集分類定義

\subsection{模型訓練}
模型訓練 模型訓練

\end{document}