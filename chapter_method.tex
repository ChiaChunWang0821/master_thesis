\documentclass[class=NCU_thesis, crop=false]{standalone}
\begin{document}

\chapter{研究方法}

\section{系統流程介紹}
本論文所開發之嬰兒危險監測系統,
其針對嬰兒影像畫面進行識別,
以判斷嬰兒是否處於危險狀態,
而須提醒照護者。

系統之完整流程為:
首先,輸入一段待觀測之嬰兒影片,
將影片切成數幀影像,
並判斷影像存在與否,
若不存在系統發出異常警告,
反之則開始對該影像畫面進行危險偵測判斷。
針對每幀嬰兒影像,
系統針對其臉部遮擋及姿勢進行辨識,
若透過模型分析為警示狀態,
則再經後續步驟判斷是否提醒照護者;
而若分析為安全狀態,
則可接續下一幀之影像進行偵測。
詳細之系統流程圖,請見\cref{fig:fig-system-flow-chart}。
\fig[1][fig:fig-system-flow-chart][!hbt]{fig-system-flow-chart.png}[系統流程圖][系統流程圖]

而本系統中,包含兩項危險辨識模型:
(1) 嬰兒臉部遮擋辨識:
先將嬰兒畫面擷取出僅含臉部範圍之影像,
再透過臉部遮擋模型判斷嬰兒臉部是否遭非奶嘴之異物遮蔽,
若是,則嬰兒為警示狀態;
(2) 嬰兒危險動作辨識:
將嬰兒影像透過嬰兒姿勢模型進行辨識,
判斷嬰兒為安全狀態:正躺或坐姿,
或為具危險性的狀態:趴躺或站立。
而若兩模型結果皆為安全,
則系統會判斷嬰兒狀態為安全,
否則,嬰兒狀態則為警示。
此二部分辨識之詳細方法,
將於3.2節及3.3節進行介紹。

\section{臉部遮擋辨識}
如前言所述,
目前醫界對於嬰兒猝死症之相關因素研究中,
注意嬰兒臉部是否遭遮蔽,
將有助於降低此症的發生。
此外,亦有研究發現嬰兒使用奶嘴,
對於預防嬰兒猝死症有幫助。

起初,
基於電腦視覺及影像處理技術,
例如:利用Cb, Cr色彩空間及ellipse clustering
~\cite{tang_hands_2008}~\cite{li_face_2011}~\cite{noauthor_python_nodate}~\cite{walkonnet_python_nodate}
等偵測膚色,
判斷嬰兒臉部是否出現非膚色之區塊,
以進行臉部遮擋辨識,
其效果如\cref{fig:fig-skin-detection}。
\fig[0.8][fig:fig-skin-detection][!hbt]{fig-skin-detection.png}[臉部膚色偵測~\cite{walkonnet_python_nodate}][臉部膚色偵測]

而後考量能有較佳的推廣性,
因此,
本研究改為使用深度學習技術,
針對嬰兒臉部影像收集資料,
進行臉部遮擋辨識模型之訓練,
並且將嬰兒使用奶嘴之情形排除,
亦即嬰兒臉部狀態共分為三種類別:
(1)面部無遮擋,為安全狀態、
(2)嬰兒正在使用奶嘴,亦為安全狀態、及
(3)嬰兒面部遭嘔吐物或毛巾等外物遮蔽,為危險狀態,而需警示照護者。

\subsection{嬰兒臉部偵測}
嬰兒臉部遮擋辨識僅需關注嬰兒臉部畫面,
故我們會先透過人臉偵測演算法進行前處理,
以獲得只涵蓋嬰兒臉部範圍之影像。
由於考量嬰兒臉部偵測之正確率及模型執行時間,
因此,
本研究選用RetinaFace~\cite{deng_retinaface_2020}
及SSD~\cite{ye_face_2021}等演算法進行此部分之處理。

\subsection{嬰兒臉部資料集}
本論文將嬰兒臉部狀態影像分為三類:
面部無遮擋、有遮擋但遮蔽物為奶嘴及有遮擋且遮蔽物非奶嘴,
前兩類判斷為安全狀態,
最後一類則為危險狀態。
對於臉部遮擋資料集之三類範例如下:

(1)臉部無遮蔽:嬰兒五官皆未被遮擋為安全狀態,如\cref{fig:fig-face-uncovered}。

(2)臉部遮蔽物為奶嘴:嬰兒正在使用奶嘴為安全狀態,如\cref{fig:fig-covered-by-pacifier}。

(3)臉部遮蔽物非奶嘴:嬰兒臉部因溢奶遭嘔吐物遮蔽,或被毛巾等其他外物遮蓋,而可能造成窒息危險,如\cref{fig:fig-covered-by-foreign-matter}。

\fig[0.6][fig:fig-face-uncovered][!hbt]{fig-face-uncovered.png}[嬰兒臉部無遮蔽][嬰兒臉部無遮蔽]
\fig[0.6][fig:fig-covered-by-pacifier][!hbt]{fig-covered-by-pacifier.png}[嬰兒臉部遮蔽物為奶嘴][嬰兒臉部遮蔽物為奶嘴]
\fig[0.6][fig:fig-covered-by-foreign-matter][!hbt]{fig-covered-by-foreign-matter.png}[嬰兒臉部遭異物遮擋][嬰兒臉部遭異物遮擋]

此嬰兒臉部資料集包含嬰兒之正臉及側臉,
共3475張照片。
我們將所有影像分為訓練、測試及驗證集,
各部分占比為70\%、20\%及10\%,
即各有2436張、697張及342張影像。

\section{危險動作辨識}
承前言所述,
除了臉部遮蔽可能造成嬰兒猝死症外,
嬰兒做出不適當的不適當也常為嬰兒逝世之原因。
例如:嬰兒側躺或趴睡時,
因頸部肌肉較弱等原因,
無力自行將臉移開,
造成呼吸困難而窒息死亡;
或者當嬰兒自行站立,
而有可能爬落嬰兒床等,
亦可能使嬰兒處於危險情境中。

\subsection{嬰兒姿勢資料集}
% 在實際情況下,
% 嬰兒姿勢多變且不固定,
% 而有些動作則需要時間資訊才得以判斷,
% 如:從正躺移至趴躺或坐姿時,
% 會做出側躺、翻身的動作;
% 從趴躺移至坐姿或站立時,
% 嬰兒的著地點有可能包含手掌、手肘、膝蓋或腳掌等。

起初,
將嬰兒姿勢分為五類:正躺、趴著、爬行、坐姿及站立,
而趴躺及爬行二類時常發生互相誤判,
致使辨識錯誤率高。
我們推測原因為此二類嬰兒皆呈現腹面朝下之姿,
而手腳位置亦有相同或相異之處,
若接續細分姿勢,
將導致動作分類過細。

因此,最終本論文將嬰兒姿勢分成基礎四類,
包含正躺(腹面朝上)、趴躺(腹面朝下)、坐姿及站立,
以供辨識嬰兒大部分之姿。
對於此四類姿勢之詳細分類定義為:

(1)正躺:嬰兒腹部面朝上,背部貼於水平面,而頭部及四肢位置不限,如
\cref{fig:fig-lie-down-on-back}。

(2)趴躺:嬰兒腹部面朝下,包含趴著或爬行等多動作,而頭部及四肢位置不限,如
\cref{fig:fig-lie-down-on-stomach}。

(3)坐姿:嬰兒臀部貼於水平面,而背部未貼於同一平面,頭部及四肢位置不限,如
\cref{fig:fig-sit}。

(4)站立:嬰兒腳掌貼於水平面,且腹部和背部皆未平行於此水平面,而頭部及上肢位置不限,如
\cref{fig:fig-stand}。

\fig[0.8][fig:fig-lie-down-on-back][!hbt]{fig-lie-down-on-back.png}[嬰兒正躺姿勢][嬰兒正躺姿勢]
\fig[0.8][fig:fig-lie-down-on-stomach][!hbt]{fig-lie-down-on-stomach.png}[嬰兒趴躺姿勢][嬰兒趴躺姿勢]
\fig[0.8][fig:fig-sit][!hbt]{fig-sit.png}[嬰兒坐姿姿勢][嬰兒坐姿姿勢]
\fig[0.8][fig:fig-stand][!hbt]{fig-stand.png}[嬰兒站立姿勢][嬰兒站立姿勢]

而為了能有較廣泛的使用情境,
所收集之嬰兒影像不限定拍攝視角,
包含俯視、平視等,
共15416張照片。
我們將所有影像分為訓練、測試及驗證集,
各部分占比為70\%、25\%及5\%,
即各有10815張、3857張及744張影像。

\subsection{危險動作判斷方法}
由於嬰兒做出趴躺及站立時,
較容易發生危險,
故當系統藉由模型辨識出嬰兒為上述兩種姿勢時,
將警示照護者須關注嬰兒狀態。

然而,
在實際情境中,
當嬰兒做出具危險性之行為時,
需持續一段時間才會導致危險的發生,
故我們不須判斷一張影像畫面為警示狀態,
就立即通知照護者。
因此,本系統使用一變數累積結果為警示之幀數,
當此變數超過一定值時,
系統才會真正發出警示提醒照護者。
此步驟不但更符合實際使用情境,
同時亦可減少因模型辨識錯誤而誤判及誤發警報的情形。

\end{document}