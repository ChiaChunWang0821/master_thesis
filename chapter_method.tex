\documentclass[class=NCU_thesis, crop=false]{standalone}
\begin{document}

\chapter{研究方法}

\section{系統流程介紹}
本論文所開發之嬰兒危險監測系統,
其針對嬰兒影像畫面進行識別,
以判斷嬰兒是否處於危險狀態,
而須提醒照護者。

系統之完整流程為:
首先,輸入一段待觀測之嬰兒影片,
將影片切成數幀影像,
並判斷影像存在與否,
若不存在系統發出異常警告,
反之則開始對該影像畫面進行危險偵測判斷。
針對每幀嬰兒影像,
系統針對其臉部遮擋及姿勢進行辨識,
若透過模型分析為警示狀態,
則再經後續步驟判斷是否提醒照護者;
而若分析為安全狀態,
則可接續下一幀之影像進行偵測。
詳細之系統流程圖,請見\cref{fig:fig-system-flow-chart}。
\fig[1][fig:fig-system-flow-chart][!hbt]{fig-system-flow-chart.png}[系統流程圖][系統流程圖]

而本系統中,危險偵測之兩核心模型辨識步驟如下:
(1) 嬰兒臉部遮擋辨識:
先將嬰兒畫面擷取出僅含臉部範圍之影像,
再透過臉部遮擋模型判斷嬰兒臉部是否遭非奶嘴之異物遮蔽,
若是,則嬰兒為警示狀態;
(2) 嬰兒姿勢辨識:
將嬰兒影像透過嬰兒姿勢模型進行辨識,
判斷嬰兒為安全姿勢:正躺或坐姿,
或為具危險性的姿勢:趴躺或站立。
而若兩模型結果皆為安全,
則系統會判斷嬰兒狀態為安全,
否則,嬰兒狀態則為危險。
此二部分辨識之詳細方法,
將於3.2及3.3進行介紹。

在實際情境中,
由於嬰兒做出危險行為時,
持續一段時間才會導致危險的發生,
因此我們不須判斷一張畫面為警示狀態,
就立即通知照護者。
故本系統使用一變數累積結果為警示之幀數,
當此變數超過一定值時,
系統才會真正發出警示,
提醒照護者須注意嬰兒之狀態。
此步驟不但更符合實際使用情境,
同時亦可減少因模型辨識錯誤而誤判的情況。

\section{臉部遮擋辨識}
如前言所述,
目前醫界對於嬰兒猝死症之相關因素研究中,
注意嬰兒臉部是否遭遮蔽,
將有助於降低此症的發生。
此外,亦有研究發現嬰兒使用奶嘴,
對於預防嬰兒猝死症有幫助。

因此,本研究會將嬰兒使用奶嘴之情形排除,
亦即將嬰兒臉部分成三種類別:
(1)面部無遮擋,為安全狀態、
(2)嬰兒正在使用奶嘴,亦為安全狀態、及
(3)嬰兒面部遭嘔吐物或毛巾等外物遮蔽,為危險狀態,而需警示照護者。

\subsection{資料集前處理}
由於此部分辨識僅關注嬰兒臉部影像,
故我們會先透過人臉偵測演算法進行前處理,
如:RetinaFace~\cite{deng_retinaface_2020}
及SSD~\cite{ye_face_2021}
等,
以獲得只涵蓋嬰兒臉部影像之資料集。

\subsection{資料集詳細介紹}
我們會將嬰兒影像分為三類:
面部無遮擋、有遮擋但遮蔽物為奶嘴及有遮擋且遮蔽物非奶嘴,
前兩類判斷為安全狀態,
最後一類則為危險狀態。
對於臉部遮擋資料集之三類範例如下:

(1)安全:嬰兒臉部五官未被遮擋,如\cref{fig:fig-face-uncovered}。
\fig[0.6][fig:fig-face-uncovered][!hbt]{fig-face-uncovered.png}[嬰兒臉部無遮蔽][嬰兒臉部無遮蔽]

(2)臉部遮蔽物為奶嘴:嬰兒正在使用奶嘴為安全狀態,如\cref{fig:fig-covered-by-pacifier}。
\fig[0.6][fig:fig-covered-by-pacifier][!hbt]{fig-covered-by-pacifier.png}[嬰兒臉部遮蔽物為奶嘴][嬰兒臉部遮蔽物為奶嘴]

(3)臉部遮蔽物非奶嘴:嬰兒臉部被嘔吐物、溢奶或其他外物遮蔽,可能造成窒息危險,如\cref{fig:fig-covered-by-foreign-matter}。
\fig[0.6][fig:fig-covered-by-foreign-matter][!hbt]{fig-covered-by-foreign-matter.png}[嬰兒臉部遭異物遮擋][嬰兒臉部遭異物遮擋]

嬰兒臉部資料集包含嬰兒之正臉及側臉,共3475張照片,
並將所有影像分為訓練、測試及預測集,
各部分占比為70\%、20\%及10\%,
即各有2436、697及342張。

\subsection{模型訓練}
使用ResNet50~\cite{he_deep_2016}
訓練臉部遮擋模型與奶嘴辨識模型,
訓練回合數皆為20。

\section{姿勢辨識}
承前言所述,
除了臉部遮蔽可能造成嬰兒猝死症外,
嬰兒做出不適當的姿勢也常為嬰兒逝世之原因,
如:嬰兒側躺或趴睡時,
因頸部肌肉較弱等原因,
無力自行將臉移開,
造成呼吸困難而窒息死亡;
或者當嬰兒自行站立,
而有可能爬落嬰兒床等,
亦可能使嬰兒處於危險情境中。

\subsection{資料集分類定義}
在實際情況下,
嬰兒姿勢多變且不固定,
而有些動作則需要時間資訊才得以判斷,
如:從正躺移至趴躺或坐姿時,
會做出側躺、翻身的動作;
從趴躺移至坐姿或站立時,
嬰兒的著地點有可能包含手掌、手肘、膝蓋或腳掌等。

本論文將嬰兒基礎姿勢分成四類,
包含了正躺(腹面朝上)、趴躺(腹面朝下)、坐姿及站立,
以供辨識嬰兒大部分之姿勢。
有別於起初將趴躺姿勢再細分為趴躺及爬行等五類,
由於動作分類過細,
導致此二類時常發生互相誤判,
致使辨識錯誤率較高。

對於四類姿勢之詳細分類定義為:

(1)正躺:嬰兒腹部面朝上,背部貼於水平面,而頭部及四肢位置不限,如
\cref{fig:fig-lie-down-on-back}。
\fig[0.8][fig:fig-lie-down-on-back][!hbt]{fig-lie-down-on-back.png}[嬰兒正躺姿勢][嬰兒正躺姿勢]

(2)趴躺:嬰兒腹部面朝下,包含趴著或爬行等多動作,而頭部及四肢位置不限,如
\cref{fig:fig-lie-down-on-stomach}。
\fig[0.8][fig:fig-lie-down-on-stomach][!hbt]{fig-lie-down-on-stomach.png}[嬰兒趴躺姿勢][嬰兒趴躺姿勢]

(3)坐姿:嬰兒臀部貼於水平面,而背部未貼於同一平面,頭部及四肢位置不限,如
\cref{fig:fig-sit}。
\fig[0.8][fig:fig-sit][!hbt]{fig-sit.png}[嬰兒坐姿姿勢][嬰兒坐姿姿勢]

(4)站立:嬰兒腳掌貼於水平面,且腹部和背部皆未平行於此水平面,而頭部及上肢位置不限,如
\cref{fig:fig-stand}。
\fig[0.8][fig:fig-stand][!hbt]{fig-stand.png}[嬰兒站立姿勢][嬰兒站立姿勢]

為了能有較廣泛的使用情境,
所收集的嬰兒影像不限定拍攝視角,
包含俯視、平視等,
共15416張照片,
並將所有影像分為訓練、測試及預測集,
各部分占比為70\%、25\%及5\%,
即各有10815、3857及744張。

\subsection{模型訓練}
使用ResNet50~\cite{he_deep_2016}
訓練嬰兒姿勢辨識模型,
訓練回合數為20。

\end{document}