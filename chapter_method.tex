\documentclass[class=NCU_thesis, crop=false]{standalone}
\begin{document}

\chapter{研究方法}

\section{系統流程介紹}
本論文所開發之嬰兒危險監測系統,
針對嬰兒影像畫面進行辨識,
以判斷其是否處於危險狀態,
而須提醒照護者。

系統之完整流程為:
首先,讀取一段待觀測之嬰兒影片,
將影片切成數幀影像,
並判斷影像存在與否,
若不存在系統發出異常警告,
反之則開始對該影像畫面進行危險偵測判斷。
針對每幀嬰兒影像,
系統對其臉部遮擋及姿勢進行辨識,
若透過模型分析為警示狀態,
則再經後續步驟判斷是否提醒照護者;
而若分析為安全狀態,
則可接續下一幀之影像進行偵測。
系統完整流程圖,請見\cref{fig:fig-flow-chart-system}。
\fig[1][fig:fig-flow-chart-system][!hbt]{fig-flow-chart-system.png}[系統流程圖][系統流程圖]

而本系統中,包含兩項危險辨識模型:
(1) 嬰兒臉部遮擋辨識:
先將嬰兒畫面擷取出僅含臉部範圍之影像,
再透過此模型判斷嬰兒臉部是否遭非奶嘴之異物遮蔽,
若是,則嬰兒為警示狀態;
(2) 嬰兒危險動作辨識:
將拍攝之嬰兒全身影像透過此模型進行辨識,
判斷嬰兒為正躺或坐姿之安全狀態,
或為需警示的趴躺及站立姿勢。
而若兩模型結果皆為安全,
則系統會判斷嬰兒狀態為安全,
否則,嬰兒狀態則為警示。
此二部分辨識之詳細方法,
將於3.2節及3.3節進行介紹。

\section{臉部遮擋辨識}
如前言所述,
目前醫界對於嬰兒猝死症之相關因素研究中,
注意嬰兒臉部是否遭遮蔽,
將有助於降低此症的發生;
另亦有研究發現嬰兒使用奶嘴,
對於預防嬰兒猝死症有幫助。
因此,
本文對於嬰兒臉部遮擋辨識將排除使用奶嘴之情境。

起初,
基於電腦視覺及影像處理技術,
例如:利用Cb, Cr色彩空間及ellipse clustering
~\cite{tang_hands_2008}~\cite{li_face_2011}~\cite{noauthor_python_nodate}~\cite{walkonnet_python_nodate}
等偵測膚色,
判斷嬰兒臉部是否出現非膚色之區塊,
以進行臉部遮擋辨識,
其效果如\cref{fig:fig-skin-detection}。
\fig[0.7][fig:fig-skin-detection][!hbt]{fig-skin-detection.png}[臉部膚色偵測~\cite{walkonnet_python_nodate}][臉部膚色偵測]

而後考量能有較佳的推廣性,
因此,
本研究改為使用深度學習技術進行臉部遮擋辨識,
針對嬰兒面部影像收集資料,
以訓練可辨識三種嬰兒臉部狀態之模型。
本部分之流程圖,請見\cref{fig:fig-flow-chart-face}。
\fig[0.9][fig:fig-flow-chart-face][!hbt]{fig-flow-chart-face.png}[臉部遮擋辨識流程圖][臉部遮擋辨識流程圖]

\subsection{嬰兒臉部偵測}
嬰兒臉部遮擋辨識僅需關注臉部畫面,
故本文會先透過人臉偵測演算法進行前處理,
以獲得只涵蓋嬰兒面部範圍之影像。

在現有人臉偵測演算法中,
偵測嬰兒臉部多有失準,
故我們在多方實驗後,
同時考量偵測之正確率及執行時間,
最終本研究選用RetinaFace~\cite{deng_retinaface_2020}
及SSD~\cite{ye_face_2021}等演算法進行嬰兒臉部偵測,
其效果如\cref{fig:fig-face-detection}。
\begin{figure}[!hbt]
    \centering
    \subcaptionbox
        {原始影像
        \label{fig:fig-face-detection-original}}
        {\includegraphics[width=0.3\linewidth]{fig-face-detection-original}}
    ~
    \subcaptionbox
        {使用RetinaFace~\cite{deng_retinaface_2020}
        \label{fig:fig-face-detection-retinaface}}
        {\includegraphics[width=0.3\linewidth]{fig-face-detection-retinaface}}
    ~
    \subcaptionbox
        {使用SSD~\cite{ye_face_2021}
        \label{fig:fig-face-detection-ssd}}
        {\includegraphics[width=0.3\linewidth]{fig-face-detection-ssd}}
    \caption{嬰兒臉部偵測結果}
    \label{fig:fig-face-detection}
\end{figure}

\subsection{嬰兒臉部資料集}
本論文將嬰兒臉部狀態分為三類,
各類定義如下:
\begin{enumerate}
    \item 臉部無遮蔽:嬰兒五官皆未被遮擋,為安全狀態,如\cref{fig:fig-face-uncovered}。
    \item 臉部遮蔽物為奶嘴:嬰兒正在使用奶嘴,為安全狀態,如\cref{fig:fig-face-covered-by-pacifier}。
    \item 臉部遮蔽物非奶嘴:嬰兒臉部因溢奶遭嘔吐物遮蔽,或被毛巾等其他外物遮蓋,而可能造成窒息危險,為警示狀態,如\cref{fig:fig-face-covered-by-foreign-matter}。
\end{enumerate}
\begin{figure}[!hbt]
    \centering
    \subcaptionbox
        {臉部無遮蔽
        \label{fig:fig-face-uncovered}}
        {\includegraphics[width=0.3\linewidth]{fig-face-uncovered}}
    ~
    \subcaptionbox
        {臉部遮蔽物為奶嘴
        \label{fig:fig-face-covered-by-pacifier}}
        {\includegraphics[width=0.3\linewidth]{fig-face-covered-by-pacifier}}
    ~
    \subcaptionbox
        {臉部遭異物遮擋
        \label{fig:fig-face-covered-by-foreign-matter}}
        {\includegraphics[width=0.3\linewidth]{fig-face-covered-by-foreign-matter}}
    \caption{嬰兒臉部資料集}
    \label{fig:fig-face-dataset}
\end{figure}

本資料集包含嬰兒之正臉及側臉共3475張照片。
我們將所有影像分為訓練、測試及驗證集,
各部分占比為70\%、20\%及10\%,
即各有2436張、697張及342張影像。

\subsection{模型訓練}
本論文使用3.2.2節之嬰兒臉部資料集,
以ResNet50~\cite{he_deep_2016}進行臉部遮擋辨識模型之訓練,
最終達成辨識三種嬰兒臉部狀態:安全、使用奶嘴或警示。

\section{危險動作辨識}
承前言所述,
除了臉部遮蔽可能造成嬰兒猝死症外,
嬰兒做出不適當的不適當也常為嬰兒逝世之原因。
例如:嬰兒側躺或趴睡時,
因頸部肌肉較弱等原因,
無力自行將臉移開,
造成呼吸困難而窒息死亡;
或者當嬰兒自行站立,
而有可能爬落嬰兒床等,
亦可能使嬰兒處於危險情境中。

本部分之流程圖,請見\cref{fig:fig-flow-chart-posture}。
\fig[1][fig:fig-flow-chart-posture][!hbt]{fig-flow-chart-posture.png}[危險動作辨識流程圖][危險動作辨識流程圖]

\subsection{嬰兒姿勢資料集}
% 在實際情況下,
% 嬰兒姿勢多變且不固定,
% 而有些動作則需要時間資訊才得以判斷,
% 如:從正躺移至趴躺或坐姿時,
% 會做出側躺、翻身的動作;
% 從趴躺移至坐姿或站立時,
% 嬰兒的著地點有可能包含手掌、手肘、膝蓋或腳掌等。

起初,
將嬰兒姿勢分為五類:正躺、趴著、爬行、坐姿及站立,
而趴躺及爬行二類時常發生互相誤判,
致使辨識錯誤率高。
我們推測原因為此二類嬰兒皆呈現腹面朝下之姿,
而手腳位置亦有相同或相異之處,
若接續細分姿勢,
將導致動作分類過細。

因此,最終本論文將嬰兒姿勢分成基礎四類,
包含正躺(腹面朝上)、趴躺(腹面朝下)、坐姿及站立,
以供辨識嬰兒大部分之姿。
對於此四類姿勢之詳細分類定義為:

(1)正躺:嬰兒腹部面朝上,背部貼於水平面,而頭部及四肢位置不限,如
\cref{fig:fig-posture-lie-down-on-back}。

(2)趴躺:嬰兒腹部面朝下,包含趴著或爬行等多動作,而頭部及四肢位置不限,如
\cref{fig:fig-posture-lie-down-on-stomach}。

(3)坐姿:嬰兒臀部貼於水平面,而背部未貼於同一平面,頭部及四肢位置不限,如
\cref{fig:fig-posture-sit}。

(4)站立:嬰兒腳掌貼於水平面,且腹部和背部皆未平行於此水平面,而頭部及上肢位置不限,如
\cref{fig:fig-posture-stand}。

\fig[0.8][fig:fig-posture-lie-down-on-back][!hbt]{fig-posture-lie-down-on-back.png}[嬰兒正躺姿勢][嬰兒正躺姿勢]
\fig[0.8][fig:fig-posture-lie-down-on-stomach][!hbt]{fig-posture-lie-down-on-stomach.png}[嬰兒趴躺姿勢][嬰兒趴躺姿勢]
\fig[0.8][fig:fig-posture-sit][!hbt]{fig-posture-sit.png}[嬰兒坐姿姿勢][嬰兒坐姿姿勢]
\fig[0.8][fig:fig-posture-stand][!hbt]{fig-posture-stand.png}[嬰兒站立姿勢][嬰兒站立姿勢]

而為了能有較廣泛的使用情境,
所收集之嬰兒影像不限定拍攝視角,
包含俯視、平視等,
共15416張照片。
我們將所有影像分為訓練、測試及驗證集,
各部分占比為70\%、25\%及5\%,
即各有10815張、3857張及744張影像。

\subsection{危險動作判斷方法}
由於嬰兒做出趴躺及站立時,
較容易發生危險,
故當系統藉由模型辨識出嬰兒為上述兩種姿勢時,
將警示照護者須關注嬰兒狀態。

然而,
在實際情境中,
當嬰兒做出具危險性之行為時,
需持續一段時間才會導致危險的發生,
故我們不須判斷一張影像畫面為警示狀態,
就立即通知照護者。
因此,本系統使用一變數累積結果為警示之幀數,
當此變數超過一定值時,
系統才會真正發出警示提醒照護者。
此步驟不但更符合實際使用情境,
同時亦可減少因模型辨識錯誤而誤判及誤發警報的情形。

\end{document}